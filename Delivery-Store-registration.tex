
\documentclass[14pt]{article}
\usepackage{xepersian} 

\settextfont{XB Zar}
\begin{document} 
\section{ فرایندهای ثبت نام پیک و فروشگاه}

اکتورها :صاحب فروشگاه - شرکت\\
نام مورد کاربرد : عقد قرارداد فروشگاه با شرکت \\
توضیح : صاحب فروشگاه از طریق مراجعه حضوری با شرکت قرارداد تنظیم می کند.\\
گام ها :\\
1- مراجعه حضوری صاحب فروشگاه برای عقد قرارداد\\
2- بیان قوانین و شرایط فروشندگی در فروشگاه آنلاین\\
3- تایید قوانین و شرایط توسط صاحب فروشگاه و بیان اطلاعات مربوط به فروشگاه خود (حجم کسب و کار، نوع کالا و …) \\
4- تایید فروشگاه آنلاین و عقد قرارداد\\
سناریو جایگزین :\\
:a-4 در صورت عدم تایید فروشگاه مراجعه کننده به شرکت، به ترتیب فرایند قرارداد یا عضویت آنها لغو می شود.\\
 
 
در سامانه ثبت نام میشود. در هنگام ثبت نام اطلاعات فروشگاه مانند آدرس و ساعات کاری پرسیده میشود. \\
 
اکتورها : شرکت- سامانه\\
نام مورد کاربرد : ثبت نام فروشگاه در سامانه \\
توضیح : شرکت فروشگاه را پس از عقد قرارداد، در سامانه ثبت نام می کند. در هنگام ثبت نام، اطلاعات فروشگاه مانند آدرس و ساعات کاری ثبت می شوند.\\
گام ها : \\
1- ثبت کردن نام فروشگاهی که عقد قراردادش تکمیل شده است در سامانه\\
2-  ثبت کردن اطلاعات فروشگاه (مانند آدرس و ساعات کاری و...) توسط شرکت در سامانه\\
سناریو جایگزین : -\\


اکتورها : پیک موتوری که در سامانه ثبت نام نشده است - شرکت \\
نام مورد کاربرد : مراجعه پیک موتوری به شرکت برای عضویت \\
توضیح : پیک های موتوری برای عضویت در سامانه و فعالیت در آن به شرکت مراجعه می کند. \\
گام ها : \\
1- مراجعه حضوری به شرکت برای عضویت \\
2- تایید شرکت  \\
سناریو جایگزین :  \\
:2-a در صورت عدم تایید پیک موتوری مراجعه کننده به شرکت، به ترتیب فرایند قرارداد یا عضویت آنها لغو می شود. \\
 
اکتورها :  شرکت - سامانه \\
نام مورد کاربرد : ثبت نام پیک موتوری در سامانه  \\
توضیح : پس از مراجعه پیک موتوری به شرکت برای عضویت، مشخصات رانندگان و وسیله نقلیه آن ها ثبت می شود. \\
گام ها : \\
1- ثبت کردن مشخصات پیک موتوری که به تایید شرکت رسیده است در سامانه \\
2-  ثبت کردن مشخصات وسیله نقلیه آن فرد در سامانه توسط شرکت .  \\
سناریو جایگزین : - \\




\end{document}
