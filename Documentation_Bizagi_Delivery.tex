\documentclass[14pt]{article}

\usepackage[a4paper, left=1.5cm, right=1.5cm, top=1.5cm, bottom=1.5cm]{geometry}

\usepackage{xepersian} 

\settextfont{XB Zar}
\begin{document} 
\section{ توضیحات مربوط به مدلسازی فرایندهای مربوط به ارسال کالا}

 
پس از تایید عملیات پرداخت، لیست خرید به سامانه ارسال می‌شود و سامانه آن لیست را به صاحب فروشگاه ارسال می‌کند. فروشگاه لیست سفارشات مشتری را آماده می‌کند. پس از آن جست‌وجوی نزدیکترین پیک موتوری آنلاین انجام می‌شود. پس ازآن، به نزدیکترین پیک موتوری یافت‌شده درخواست داده می‌شود. پاسخ پیک موتوری در دوحالت رخ می‌دهد که برای بررسی آن باید از گیت XOR استفاده نمود تا حالت‌های احتمالی که تنها یکی از آن‌ها می‌تواند رخ دهد، تعیین شود:\\
\begin{flushright}
\begin{itemize}
\item حالت اول : اگر پیک موتوری درخواست را برای ارسال سفارش تایید کند، جست‌و‌جو مجددا برای یافتن نزدیک‌ترین پیک موتوری آنلاین انجام می‌شود.\\
\item حالت دوم : اگر پیک موتوری درخواست را تایید کند، وظیفه بعدی انجام می‌شود.\\
\end{itemize}
\end{flushright}

پس از مشخص شدن پیک موتوری، اطلاعات خرید به او ارسال می‌شود. سپس پیک موتوری با مراجعه به فروشگاه سفارش مشتری را از فروشگاه دریافت نموده و سپس کالاهای سفارشی را به وی تحویل می‌دهد. سپس مشتری می‌تواند نظر خود را در ارتباط با کالای سفارش داده‌شده و یا پیک موتوری ارسال نماید.\\


\end{document}