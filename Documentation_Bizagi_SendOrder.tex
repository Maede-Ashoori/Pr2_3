
\documentclass[14pt]{article}

\usepackage[a4paper, left=1.5cm, right=1.5cm, top=1.5cm, bottom=1.5cm]{geometry}

\usepackage{xepersian} 

\settextfont{XB Zar}
\begin{document} 
\section{ توضيحات مربوط به مدلسازی فرایندهای مربوط به ایجاد سفارش}




*فرض : مشتری در هنگام ثبت‌نام، نام کاربری و رمزعبوری برای خود تعیین می‌کند و در هنگام ورود حساب کاربری، وی باید اطلاعات کاربری خود را (نام کاربری ورمزعبور) را وارد کند تا وارد حساب کاربری شود.\\

مشتری برای ارسال سفارش ابتدا وارد سامانه شده و اطلاعات کاربری خود را که پیشتر در هنگام ثبت‌نام تعیین نموده  در صفحه ورود ثبت می‌کند و دکمه ورود را فشار می‌دهد. \\
حال صحت اطلاعات کاربری باید بررسی شود؛ به همین دلیل از گیت\LTRfootnote{\label{myfootnote}gate}    
XOR  استفاده می‌شود تا حالت‌های احتمالی که تنها یکی از آن‌ها می تواند رخ دهد، تعیین شود.\\
\begin{flushright}
\begin{itemize}
\item حالت اول : اگر اطلاعات مشتری صحیح نباشد، مشتری باید مجددا اطلاعات کاربری را ثبت کند؛ درنتیجه پیغامی به مشتری داده می‌شود و این شاخه به XOR قبل از وظیفه وارد کردن اطلاعات کاربری ختم می‌شود.\\
\item حالت دوم : اگر اطلاعات کاربری مشتری صحیح باشد که در این حالت مشتری وظیفه بعدی را انجام می‌دهد. \\
\end{itemize}
\end{flushright}

پس از تایید ورود مشتری، سامانه لیست فروشگاه‌های نزدیک را نشان می‌دهد و سپس مشتری فروشگاه موردنظر خود را انتخاب می‌کند. سامانه لیست کالاهای فروشگاه انتخاب‌شده و اطلاعات آنها را به مشتری نمایش می‌دهد. مشتری کالاهای مورد نظر و تعداد آنها را در سامانه ثبت می‌کند. پس از تکمیل سبد خرید، سامانه لیست خرید را نشان می‌دهد. حال سامانه برای تایید سبد خرید، دو موضوع را بررسی می‌کند: سطح موجودی کالا و ساعت کاری فروشگاه انتخاب‌شده. هر دو بررسی باید انجام شود؛ درنتیجه در اینجا از گیت AND استفاده می‌شود:\\
\begin{flushright}
\begin{itemize}
\item بررسی اول : در این شاخه، سطح موجودی بررسی می‌شود. در این بررسی، دو حالت رخ می‌دهد؛ به همین دلیل از گیت XOR  استفاده می‌شود تا حالت‌های احتمالی که تنها یکی از آن‌ها می تواند رخ دهد، تعیین شود.\\
\begin{flushright}
\begin{itemize}
\item حالت اول: اگر موجودی کالا تمام شده باشد، در این حالت، مشتری باید به لیست فروشگاه‌ها برگردد تا کالای جایگزینی را انتخاب کند. \\
\item حالت دوم : اگر کالا موجودی داشته‌ باشد، در این حالت، باید سایر کالاهای موجود در سبد بررسی شود که دو حالت رخ می‌دهد :\\
\begin{flushright}
\begin{itemize}
 \item حالت اول : اگر موجودی همه کالا ها بررسی شده باشد و اگر زمان سفارش در محدوده ساعت کاری فروشگاه باشد، سبدخرید تایید می‌شود. در غیر اینصورت تا وقتی که زمان کاری فروشگاه فرا نرسد، سبد خرید تایید نمی‌شود.\\
\item حالت دوم : اگر چند کالا در سبد خرید موجود باشد و موجودی همه کالاها بررسی نشده باشد، شاخه این حالت به XOR قبل از XOR بررسی موجودی ختم می‌شود؛ زیرا وقتی این حالت فعال باشد به این معنی است که کالایی وجود دارد که موجودی آن چک نشده‌است، پس به گیت XOR بررسی موجودی ختم می‌شود.\\

\end{itemize}
\end{flushright}
\end{itemize}
\end{flushright}

\item بررسی دوم :در این شاخه، ساعت کاری فروشگاه بررسی می‌شود. در این بررسی، دو حالت رخ می‌دهد؛ به همین دلیل از گیت XOR  استفاده می‌شود تا حالت‌های احتمالی که تنها یکی از آن‌ها می‌تواند رخ دهد، تعیین شود.\\
\begin{flushright}
\begin{itemize}
\item حالت اول : اگر زمان سفارش در محدوده ساعت کاری فروشگاه نباشد، سامانه سبد خرید را نشان می‌دهد و مشتری باید در زمان دیگری برای سفارش اقدام کند. درنتیجه این شاخه به XOR قبل از وظیفه نشان دادن سبد خرید منتهی می‌شود.\\
\item حالت دوم : اگر زمان سفارش در محدوده ساعت کاری فروشگاه باشد و اگر موجودی همه کالا ها بررسی شده باشد و همه کالاهای سبد خرید موجودی داشته باشند، سبد خرید تایید می‌شود در غیر این صورت مشتری باید سبد خرید خود را به‌روزرسانی کند.\\
\end{itemize}
\end{flushright}
\end{itemize}
\end{flushright}
پس از بررسی و تایید زمان سفارش و موجودی کالا، مشتری پرداخت را انجام می‌دهد. در هنگام پرداخت، دو حالت رخ می‌دهد که باید از گیت XOR استفاده کرد تا حالت‌های احتمالی که تنها یکی از آن‌ها می‌تواند رخ دهد، تعیین شود:\\
\begin{flushright}
\begin{itemize}
\item حالت اول : اگر عملیات پرداخت موفقیت‌آمیز نباشد، سامانه سبد خرید را نمایش می‌دهد.\\
\item حالت دوم : اگر عملیات پرداخت موفقیت‌آمیز باشد، وظیفه بعدی انجام می‌شود.\\

\end{itemize}
\end{flushright}
\end{document}