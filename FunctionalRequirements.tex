\documentclass[12pt]{report}
\usepackage{xepersian}
\settextfont{XB Zar}

\begin{document}
\section{نیازمندی های کاربردی }
نیازمندی های کاربردی را به طور کلی به سه دسته میتوان تقسیم نمود:\\
-الزامات مدیریت: شامل اقداماتی که کارمندان و کارکنان توسط سیستم انجام می دهند. \\
-الزامات بازاریابی : انواع فعالیت های بازاریابی که وبسایت از آن ها پشتیبانی می کند. \\
-الزامات مرتبط با فروش : اقداماتی که تیم مدیریت و فروش باید انجام دهد.\\
برخی از مهم ترین نیازمندی های کاربردی از دسته های مختلف به همراه توضیحات هریک در ادامه آورده می شود:\\
دسترسی به پنل مدیریت: از بین کارمندان با سطوح دسترسی معین، برخی بایستی به پنل مدیریت دسترسی داشته و بتوانند تغییرات لازم را از آن طریق اعمال کنند و موارد نیازمند پشتیبانی را پیگیری نمایند.\\
بررسی سابقه سفارش : کارمندان بخش های مختلف به خصوص فروش و مدیران این بخش نیاز دارند تا بتوانند سابقه فروش و ارقام و سایر جزئیات مرتبط با آن را در قالب داشبوردهای مدیریتی و اطلاعات با قالب های مختلف مشاهده کنند و در تصمیم گیری درخصوص سیاست ها و نیز کارمندان خود با اطلاعات کافی عمل کنند.\\
ایجاد فاکتور: پس از ثبت سفارش موفق کاربران (مشتریان) لازم است نسخه الکترونیکی فاکتور خرید به وی نمایش داده شود تا در صورت لزوم تغییرات لازم را پیش از پرداخت نهایی در آن ایجاد کند.\\
تغییر قیمت ها : افراد مرتبط با قیمت گذاری شامل مدیران سطح بالا و برخی مدیران میانی نیازمند ساختاری جهت تغییر قیمت ها در صورت لزوم هستند.\\
مدیریت طبقه بندی کالاها : همانطور که پیش تر ذکر شد، طبقه بندی کالاها تاثیر فوق العاده ای در انتقال یک تجربه خوب کاربری و حفظ کاربر در سایت دارد. به این منظور باید افراد مرتبط، امکان تغییر و به روزرسانی طبقه بندی کالاهای مختلف در سایت را داشته باشند.\\
ایجاد تخفیف و کد تبلیغاتی : برای افراد بخش بازاریابی در اختیار داشتن امکاناتی  برای ایجاد کد تخفیف و کارت هدیه با توجه به مناسبت های مختلف و برمبنای مشتریان و اطلاعات تاریخچه ای آن ها که در نرم افزارهای CRM و باشگاه مشتریان قرار دارد، اهمیت زیادی دارد.\\
تغییر قوانین قیمت سبد خرید: وجود امکانات خودکار سازی جریان کار جهت اعمال تخفیفات بر روی سبد خرید برمبنای قوانین مشخص یک نیازمندی مهم دیگر است.\\
 به حداقل رساندن مراحل خرید: این مورد بالاترین اولویت را در بین نیازهای فروش داراست و  به ویژه باید اطمینان کسب نمود که روند پرداخت در سریع ترین زمان ممکن انجام می شود.علاوه بر این ، نماد سبد خرید باید به وضوح در هر صفحه از سایت قابل مشاهده باشد. این کار تعامل خریدار بالقوه را برای کار با سبد خرید آسان تر می کند (مشاهده کالاهای انتخاب شده، حذف موارد غیر ضروری، تنظیم مقدار واحد محصول موردنیاز و ...)\\
عملکرد مناسب در پلتفرم های تلفن همراه و تبلت: امروزه ، تعداد بسیار زیادی از کاربران اینترنت خریدهای آنلاین را از طریق تلفن های هوشمند و تبلت ها انجام می دهند تا از طریق لپ تاپ و رایانه های شخصی؛ لذا این مورد یکی از الزامات کاربردی هر کسب وکار آنلاین می باشد.\\
طٰراحی منحصر به فرد و متمایزکننده: ویژگی دیگری که یک وبسایت یک کسب وکار آنلاین را تعریف می کند ، طراحی منحصر به فرد و معتبر آن است.نیاز به شخصی سازی وبسایت یک مولفه مهم در طراحی آن است.\\
مطالب مرتبط و مفید: قراردادن محتوا و مقالات تخصصی مرتبط با کسب وکار و محصولات علاوه بر مزیت بهبود SEO سایت، راهی برای تأمین نیازهای دقیق ترین مشتریانی است که به دنبال چیزی بیش از توصیف استاندارد محصول هستند.\\
ابزارهای خبرنامه ایمیل:خبرنامه از جمله موارد ضروری بازاریابی است که لازمه وبسایت هر کسب وکار آنلاین است و  این امکان را فراهم میکند تا بیشتر مراحل تعامل فعال با مشتریان فروشگاه آنلاین از طریق ایمیل خودکار شود.\\ 
یکپارچگی با شبکه های اجتماعی: امکان اتصال به  شبکه های اجتماعی محبوب از طریق API و اختصاص بخشی از وبسایت برای رتبه بندی و افرودن نظرات کاربران در آن دو نیازمندی مهمند که باید مدنظر قرار گیرند. این  موضوع میزان اعتماد کاربر به سیستم تجارت را کسب وکار را افزایش می دهد و نیاز وی را برای ارزیابی محصول برآورده می کند.\\
یکپارچه سازی سیستم های حمل و نقل و پرداخت: این موضوع که کاربران برای پرداخت ها و نحوه ارسال کالای انتخابی(سیستم حمل ونقل) بتوانند از میان چند گزینه ترجیحات خود را برگزینند، اهمیت زیادی دارد و لذا سیستم باید با سیستم های حمل و نقل و پرداخت متفاوتی یکپارچه شود.\\
گفت وگوی برخط: گفت وگوی برخط نکته ای است که باید در میان الزامات اساسی مدیریت مورد توجه قرار گیرد. هر چقدر محصول ساده باشد ، داشتن این امکان در سایت که مشتریان می توانند توسط آن بلافاصله پاسخ سوالات خود را دریافت کنند ، بسیار مطلوب است و توصیه می شود برخی از الگوریتم های ساده را برای یک بات \footnote{\label{myfootnote}bot}گفت و گو شبیه سازی شود که این امر باعث کاهش تعداد تماس ها با اپراتورهای زنده می شود.\\
مجموعه ای مفید از فیلترها:هرچه امکانات جست و جوی پیشرفته بیشتری برای وبسایت فراهم شود، سبب بهتر یافتن محصولات توسط کاربران و خرید بیشتر و در درازمدت سود بیشتر کسب و کار می شود.
\end{document}