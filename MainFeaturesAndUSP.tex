
\documentclass[14pt]{article}
\usepackage{xepersian} 

\settextfont{XB Zar}
\begin{document} 
\section{ویژگی های اصلی و منحصر به فروش فروشگاه آنلاین}

ویژگی های اصلی و الزامی فروشگاه های آنلاین :

\begin{flushright}
\begin{enumerate}

\item منوی پیمایشی با دسته بندی های مشخص و دقیق جهت جست وجوی محصولات مختلف
\newline
وجود منوی پیمایشی در هر فروشگاه انلاین ضروری به نظر می رسد. آنچه اهمیت دارد، آن است که در ارتباط با منوی پیمایشی دسته بندی های دقیق و شفاف محصولات براساس معیارهای صحیح تاثیر به سزایی در جذب مشتریان و تشویق آنان به ادامه خرید دارد. به طورکلی ساختار وبسایت علاوه بر این که از لحاظ فنی باید از کیفیت مناسبی برخوردار باشد، باید محیط ساده و کاربرپسندی نیز داشته باشد تا عموم کاربران در استفاده از آن مشکلی نداشته باشند. 
\item توضیحات دقیق در مورد هر محصول
\newline
ذکر توضیحات مفصل و با جزئیات کالا در ارتباط با تمامی محصولات یک فروشگاه آنلاین ضروری است. این امر در مورد کالاهای مشابه که تفاوت اندکی دارند،‌بسیار اهمیت دارد.(درارتباط با فروشگاه مورد بررسی ذکر جزئیاتی همچون جنس محصول، طرح،سایز، نام تولیدکننده،مورد استفاده محصول و…)به طور کلی ویژگی های هر محصولی باید به طور دقیق ذکر شود اما به گونه ای که کاربر در کمترین زمان بیشترین اطلاعات را به دست آورد.
\item تصاویر باکیفیت و جزئیات از محصولات
\newline
ارائه تصاویر با کیفیت و از زوایای مختلف از محصول (در مورد پوشاک:تا حد امکان ارائه تصویری از تن خور آن محصول) به همراه قابلیت بزرگنمایی اهمیت بسیاری دارد.
\item ویدیوهای معرفی محصولات و ویژگی های آنان
\newline
ارائه جزئیات دقیق از محصول به صورت سه بعدی در قالب ویدیو در آگاهی کاربران از محصول انتخابی اهمیت زیادی داشته و کاربران را در وضعیت تصور استفاده از محصول قرار داده و آن ها را به خرید آن محصول ترغیب می کند.
\item بخش نظرات کاربران
\newline
وجود این بخش در وبسایت هم برای کاربران و هم کسب وکار اهمیت دارد. مزیت وجود این بخش برای کاربران در آگاهی از نظرات کاربرانی است که تجربه استفاده از محصول را داشته اند و به این ترتیب کاربران اعتماد بیشتری درمورد انتخاب یا عدم انتخاب آن محصول پیدا می کنند. برای کسب و کار نیز نظرات منفی می توانند بسیار سودمند باشند؛‌ هرچند که ممکن است وجود این بخش در وبسایت به لحاظ نظرات منفی کاربران با ریسک همراه باشد اما به هر شکل اطلاعات مفیدی در ارتباط با قطع همکاری با تامین کنندگانی که کالای آن ها نارضایتی بیشتری در پی دارد، فراهم می کند و امکان کار بر روی نقاط ضعف را فراهم می کند.
\item صفحه جذاب و حاوی اطلاعات«درباره ما» با جزئیات تماس
\newline
ذکر اطلاعات کلی کسب وکار همچون تاریخچه مختصر، صاحبان کسب وکار، مکان دفتر مرکزی و راه های ارتباطی در صفحه درباره ما سبب جلب اعتماد کاربران شده و به بازدیدکنندگان اطمینان می بخشد که با افراد حقیقی در واقع در تعاملند. به علاوه به برند کسب و کار نیز چهره و شخصیت می بخشد.
\item بخش سوالات متداول و پاسخ به آنان
\newline
وجود این بخش نه تنها باعث می شود مشتریان بهتر آگاه شوند بلکه سبب صرفه جویی در وقت تیم فروش کسب وکار نیز می شود. در ارتباط با این بخش باید اطمینان حاصل شود که پاسخ ها به وضوح و مختصر نوشته شده اند تا کاربران پاسخ سوالات خود را به راحتی بیابند. نمونه ای از این سوالات در کسب و کار مورد بررسی عبارتنداز: چه زمانی طول می کشد تا محصول ارسال شود؟ هزینه ارسال محصول چقدر است؟ راه های مختلف پرداخت کدامند؟ روال مرجوعی کالا به چه صورت است؟ و...
\item عملکرد خوب و مناسب بر روی پلتفرم تلفن همراه
\newline
بخش عمده کاربران کسب وکارهای آنلاین امروزه از طریق تلفن همراه با آن ها ارتباط برقرار می کنند؛ بنابراین عملکرد مناسب وبسایت و در دسترس بودن تمام قابلیت های آن بر روی تلفن همراه درست مانند کار با وبسایت بر روی دسکتاپ اهمیت دارد و در صورت عدم توجه به این امر بخش عمده ای از کاربران نادیده گرفته می شوند.
\item امکان پرداخت از طرق مخلتف
\newline
کاربران مختلف روش های پرداخت مختلفی را ترجیح می دهند؛ برخی روش های سنتی و درمحل را ترجیح می دهند، درحالی که بسیاری روش های آنلاین پرداخت را انتخاب می کنند. توجه به دسته های مختلف کاربران و فراهم کردن روش های پرداخت متفاوت و درگاه های مختلف پرداخت،‌ اهمیت دارد.
\item امکان ردیابی سفارشات در طی مراحل مختلف از آغاز سفارش تا تحویل
\newline
امکان ردیابی، مشتریان را قادر می سازد زمان تحویل محصول را پیش بینی کنند، این امر خصوصاً در مورد پرداخت وجه نقد هنگام تحویل مناسب است. همچنین آگاهی از مراحل مختلفی که محصول طی می کند، در ایجاد حس امنیت در مشتری تاثیرگذار است.
\item امکان گفت و گوی زنده با تیم فروش هنگام بروز مشکل یا ایجاد سوال
\newline
فراهم کردن چنین امکانی برای رفع سوالات و مشکلات و چالش های کاربران حین مسیر خرید و همچنین قبل و بعد از آن اهمیت زیادی دارد؛ اما نکته حائز اهمیت آن است که این گفت وگو و پاسخ و خدمت رسانی به مشتری باید با سرعت بسیار خوبی انجام شود و درغیر اینصورت ارزش افزوده مدنظر را ایجاد نمی کند.
\item حضور فروشگاه در شبکه های اجتماعی و فعالیت در آن ها
\newline
امروزه بخش مهمی از مشتریان کسب وکارهای آنلاین از طریق شبکه های اجتماعی فراهم می شوند و علاقه مندند که از طریق پلتفرم های مختلفی با کسب وکار در ارتباط باشند؛‌ لذا حضور در شبکه های اجتماعی برای برقراری ارتباط موثر با مشتریان فعلی و نیز برگزاری انواع کمپین تبلیغاتی و … برای جذب مشتریان جدید اهمیت دارد.
\item امکان تشکیل لیست علاقه مندی ها 
\newline
لیست علاقه مندی ها به مشتریان این امکان را می دهد تا تمام محصولاتی را که دوست دارند در یک مکان ذخیره کنند. بعضی از آن ها ممکن است در آینده توسط خود مشتریان خریداری شوند. فراهم کردن این امکان و به کارگیری روش های یادگیری ماشین و داده کاوی این لیست ها می تواند سبب به دست آوردن دید بهتر از مشتری و ارائه پیشنهادات موثری به وی شود که می تواند برای کسب وکار ارزش ایجاد کند.


\end{enumerate}
\end{flushright}

\begin{flushright}
ویژگی های منحصر به فروش فروشگاه آنلاین مورد بحث:
\begin{enumerate}
\item فروش در شهر تهران
\newline
موقعیت فیزیکی یک مزیت بزرگ برای کسب وکار محسوب می شود. در شهر تهران جمعیت 5.17 درصد جمعیت کل کشور را پوشش می دهد که حائز اهمیت است و سبب ایجاد بازار محلی گسترده ای می شود و به علت جمعیت بیشتر تعداد نیروی کار نیز زیاد است (به خصوص که از شهرهای اطراف تهران نیز برای کار به تهران می آیند). همچنین دسترسی به اینترنت و گوشی های هوشمند و پلتفرم های ارتباطی در این شهر بیشتر از سایر شهرهاست که سبب افزایش مشتری بالقوه بیشتر می شود. 

\item قیمت متفاوت و به صرفه پوشاک برند
\newline
رقابت به واسطه قیمت گذاری مناسب پوشاک برند میتواند یک مزیت بزرگ فروشگاه آنلاین است. در کسب وکار آنلاین مورد بحث، پوشاک برند به صورت انبوه و بدون واسطه خریداری می شود. به اینصورت که در هر دوره، افرادی به عنوان خریدار به محل های مختلف (داخل کشور یا خارج) فرستاده می شوند و به طورمستقیم از تولیدی یا فروشگاه های برندهای معتبر خرید انجام می دهند که این سیاست در قیمت گذاری مناسب اجناس نقش بسزایی دارد.

\item فروش اجناس خاص
\newline
به عنوان مثال، فروشگاه آنلاین مورد بحث قرارداد های مختلفی با برند های ایرانی در حوزه صنعت بافندگی و یا چرم دارد و این برند ها انواع لباس های بافتنی (دستباف) و چرم (طبیعی یا مصنوعی مرغوب) را تولید می کنند که بسیار خاص و کمیاب می باشند.
\item قابلیت های ویژه برای اطمینان از کیفیت و طراحی لباس ها
\newline
یکی از بزرگترین دغدغه های خرید لباس این است که بعد از شستشو، کیفیت لباس تغییری نکند. فروشگاه مورد بحث، امکان مرجوعی بعد از شست و شو را فراهم می کند که اگر با یک بار شستشوی لباس خریداری شده، کیفیت لباس افت کرد، مشتری می تواند لباس خریداری شده را بازگرداند. این اقدام نشان می دهد که فروشگاه به کیفیت محصولات خود اطمینان دارد و این اطمینان را به مشتریان خود نیز انتقال می دهد.
دغدغه دیگر مشتریان هنگام خرید لباس این است که در وهله اول عیب های طراحی لباس مشاهده نمی شود. در نتیجه فیلم هایی در محل ها و موقعیت های آب وهوایی مختلف توسط فردی که لباس را پوشیده است، گرفته می شود که به تصمیم گیری خرید مشتری کمک کند.

\item قابلیت ویژه برای مشتریان وفادار
\newline
فروشگاه مورد بحث، برای مشتریان با رتبه طلایی در باشگاه مشتریان، امکان اتوشویی رایگان لباس های مجلسی و خاص را فراهم می کند که مشتریان می توانند لباس خود درب منزل تحویل دهند و تحویل بگیرند.(البته فروشگاه خود این عمل را انجام نمی دهد؛ بلکه این کار به صورت برون سپاری و با همکاری با کسب وکارهای با درصد اطمینان بالا در این حوزه انجام می شود)

\item فروش به صورت آنلاین از طریق پلتفرم های مختلف
\newline
مشتریان دوست دارند به روش های مختلف خرید کنند. Omnichannel به معنای دیدار با مشتریان خود در جایی است كه آنها هستند. حضور در پلتفرم های مختلف از جمله وبسایت، صفحه اینستاگرام، کانال تلگرامی و اپلیکیشن، سبب میشود که راه دسترسی مشتری به فروشگاه ما آسان شود.

\item استفاده از نرم افزار CRM 
\newline
این نرم افزار به کسب و کار کمک میکند تا مدیریت کانال های مختلف در ارتباط با مشتری به خوبی انجام شود. پاسخگویی به موقع به مشتریان یکی از عوامل موثر در فروش و رضایت مشتری است که نرم افزار CRM در این امر به تیم فروش کمک می نماید.

\item پرو کردن قبل از خرید
\newline
در خرید آنلاین برخلاف خرید حضوری این امکان فراهم نیست که قبل از خرید، لباس پوشیده شود و در خرید لباس اندازه، رنگ، طرح بسیار اهمیت دارد. در فروشگاه آنلاین مورد بحث این قابلیت فراهم شده است که مشتریان با رتبه طلایی و نقره ای می توانند ( با دریافت کارت معتبری به عنوان اطمینان برای پس دادن لباس ها) 5 لباس را انتخاب کنند و قبل از خرید، لباس را امتحان کنند.


\item گزینه اهدا به امور خیریه
\newline
فروشگاه مورد بحث با موسسات امور خیریه (بیماری های خاص مانند: سرطان، اتیسم و …) قرار می بندد و در مناسبت های خاص مانند روز جهانی اتیسم، طرح خیریه ای را در پلتفرم خود ایجاد می کند که مشتریان می توانند لباسی را برای امور خیریه خریداری کنند که فروشگاه مبلغ پرداختی و لباس خریداری شده را به عنوان اهدایی آن فرد به موسسات امور خیریه اهدا می کند.
\end{enumerate}
\end{flushright}
\end{document}
