\documentclass[12pt]{report}
\usepackage{xepersian}
\settextfont{XB Zar}

\begin{document}
\section{تنظیم جدول فرایندهای آماده سازی سفارشات و انتخاب پیک تا تحویل محصول به مشتری}

اکتورها :صاحب فروشگاه- سامانه\\
نام مورد کاربرد : آماده سازی سفارشات توسط فروشگاه\\
توضیح : لیست سفارشات مشتری به صاحب فروشگاه نشان داده می شود تا آماده سازی سبد خرید صورت گیرد. \\
گام ها :\\
1- ارسال لیست سفارشات مشتری به صاحب فروشگاه\\
2- اماده سازی لیست سفارشات مشتری توسط فروشگاه\\
سناریو جایگزین : -\\
اکتورها : پیک ثبت نام شده و آنلاین - سامانه-مشتری\\
نام مورد کاربرد : ارسال سفارش توسط پیک موتوری مناسب\\
توضیح : همچنین در سامانه بر روی پیک های موتوری آنلاین و در دسترس جستجویی انجام شده تا نزدیکترین موتور برای ارسال سفارش انتخاب شود. در صورت تایید پیک برای قبول سفارش، اطلاعات خرید اعم از آدرس فروشگاه، آدرس مقصد (آدرس مشتری) و لیست خرید به پیک موتوری ارسال میشود تا فرآیند تحویل سفارش آغاز گردد. در صورتی که پیک درخواست را رد نمود، جستجوی دیگری صورت میگیرد تا پیک دیگری پیدا شود(و این روند تا پیدا شدن پیک ادامه مییابد.) پیک انتخاب شده، خرید را به مشتری مورد نظر تحویل می دهد.\\
گام ها :\\
1- جستجو در پیک های موتوری آنلاین و در دسترس\\
2- انتخاب نزدیکترین پیک موتوری آنلاین توسط سامانه \\
3- تایید درخواست سامانه توسط پیک موتوری\\
4- ارسال اطلاعات خرید به پیک موتوری توسط سامانه\\
5- تحویل خرید مشتری\\
سناریو جایگزین : \\
3-a :پیک موتوری درخواست سامانه را رد کند. در این حالت جستجوی دیگری صورت میگیرد تا پیک دیگری پیدا شود(و این روند تا پیدا شدن پیک ادامه مییابد.)
اکتورها : خریدار- سامانه\\
نام مورد کاربرد : ثبت نظرات و امتیازات توسط مشتری در سامانه\\
توضیح : پس از تحویل سفارش به مشتری، مشتری امکان ارسال نظر و امتیاز به فروشگاه لباس و پیک موتوری را خواهد داشت. این امتیازها و نظرات برای فروشگاه لباس و پیک موتوری ذخیره شده و امکان مشاهده آن در آینده وجود خواهد داشت. \\
گام ها :\\
1-  ورود مشتری ثبت نام شده‌ به سامانه\\
2 - ورود به حساب کاربری\\
3- ارسال نظرات و امتیاز توسط مشتری خریدار\\
4- ذخیره نظرات و امتیازها توسط سامانه\\
سناریو جایگزین : -
 \end{document}
 
