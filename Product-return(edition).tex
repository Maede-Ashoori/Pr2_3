
\documentclass[14pt]{article}
\usepackage{xepersian} 

\settextfont{XB Zar}
\begin{document} 
\section{ فرایندهای بازگشت محصول}
 
اکتورها : خریدار- تیم پشتیبانی - سامانه - صاحب فروشگاه    \\
نام مورد کاربرد : درخواست مرجوعی کالای خریداری شده\\
توضیح : چنانچه خریدار به هر دلیلی قصد مرجوع کردن کالا را داشته باشد، درخواست خود را در سامانه ثبت می کند. ثبت درخواست مرجوعی کالا توسط مشتری مورد بررسی تیم پشتیبانی قرار می گیرد و در صورت برقراری شرایط مرجوعی به تایید آن ها می رسد و در غیراینصورت درخواست لغو می شود.\\
گام ها :\\
1- ورود به سامانه\\
2- ورود به حساب کاربری\\
3-  تماس با تیم پشتیبانی جهت درخواست مرجوع کردن کالا\\
4- تایید درخواست موجوعی کالای خریداری شده توسط تیم پشتیبانی\\
سناریو جایگزین : \\
:4.a عدم تایید تیم پشتیبانی.در این حالت درخواست مرجوعی لغو می شود.\\


اکتورها : پیک موتوری - خریدار\\
نام مورد کاربرد :مرجوع کردن کالای خریداری شده\\
توضیح :  در صورت تایید تیم پشتیبانی یک پیک موتوری مسئول می شود تا آن کالا را بازگرداند.\\
گام ها :\\
1- دریافت مشخصات خریدار (آدرس، شماره تلفن و …)از فروشگاه\\
2- دریافت کالای مرجوعی از مشتری\\
3- تحویل کالای مرجوعی به فروشگاه مبدا\\
سناریو جایگزین : -\\


\end{document}
