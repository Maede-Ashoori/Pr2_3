


\documentclass[14pt]{article}
\usepackage[a4paper, left=1.5cm, right=1.5cm, top=1.5cm, bottom=1.5cm]{geometry}

\usepackage{xepersian} 

\settextfont{XB Zar}

\begin{document}

\begin{flushright}
\begin{itemize}
\section{ چشم‌انداز محصول}

\subsection{فرضیات}

\item این کسب‌و‌کار با فروشگاه‌هایی همکاری می‌کند که اجناس آن‌ها بازه قیمتی متوسط و روبه‌بالا دارند. (پوشاک با سطح کیفیت متوسط و بالا و اجناس برند و لوکس)
\item رده سنی کاربران هدف، شامل تمام رده‌ها :‌کودک، نوجوان، جوان و بزرگسال بوده و برای هر دو جنسیت مونث و مذکر می‌باشد.
\item همکاری با فروشگاه‌های پوشاک در سرتاسر شهر تهران صورت می‌گیرد و محدود به همین شهر است.

\subsection{مشتریان}


برای یک فروشگاه آنلاین پوشاک دسته‌های مختلفی از کاربران (مشتریان) که با فروشگاه در تعاملند،‌ قابل تصور است. مشتریان کسب‌و‌کار فرضی مد نظر براساس فرضیات ذکرشده به بخش‌های زیر تقسیم می‌شوند:
\item از نظر جنسیت: مخاطبان کسب‌و‌کار شامل هر دو جنسیت مونث و مذکر می‌باشند.
\item از نظر سن:‌ مخاطبان کسب‌و‌کار شامل همه رده‌های سنی (کودک و نوجوان، جوان و بزرگسالان) می‌باشند.
\item از نظر جغرافیایی: ساکنین تهران و شهرهای نزدیک که امکان ارسال کالا با پیک برای آن‌ها فراهم است. (به عنوان مثال:‌کرج)
\item از نظر رفتاری‌: مشتریانی که برای کیفیت خوب کالا ارزش قائلند و به علاوه اعتماد به خرید آنلاین دارند و توانایی انجام این نوع خرید در آن‌ها وجود دارد.


\subsection{ ذینفعان}
ذینفعان یعنی افراد و گروه‌هایی که به نوعی در موفقیت یا شکست یک کسب‌وکار، سهیم هستند که برای کسب‌وکار مدنظر شامل موارد زیر می‌باشند:\\
\item مشتریان کسب‌و‌کار‌ \\
مشتریان برای تهیه کالا یا خدمات به شرکت وابسته هستند. آن‌ها با هر خریدی که انجام می‌دهند از کسب‌وکار پشتیبانی می‌کنند و همچنین هر خرید آن‌ها به کسب‌وکار نشان می‌دهد که در چه محصولات و خدماتی سرمایه‌گذاری بیشتری انجام دهد. با انجام این کار، مشتریان به راهنمایی جهت یک تجارت کوچک کمک می‌کنند.
\item صاحب کسب‌وکار (مدیرعامل)\\
صاحب کسب‌وکار کنترل نهایی بر شرکت دارد و کلیه تصمیمات راهبردی و تصمیم‌گیری نهایی در ارتباط با تفویض وظایف و … برعهده اوست.\\
\item تیم IT \\
 اعضای تیم که در راه‌اندازی سایت و اپلیکیشن کسب‌وکار وخدمات مربوط به آن و کلیه خدمات مرتبط به فناوری اطلاعات نقش دارند.\\
مدیر اجرایی IT که نقش مهمی در پروژه تجارت الکترونیکی و رهبری آن دارد و نگرانی‌های فناوری اطلاعات را به مجموعه منتقل می‌کند. \\
\item تیم فروش و بازاریابی\\
تیم فروش نقش به سزایی در موفقیت کسب‌وکار ایفا می‌کند؛ چرا که در واقع کسب کارآمد درآمد به این گروه وابسته است. در ارتباط با کسب‌وکارهای آنلاین استفاده مناسب از فناوری‌ها و سیستم‌های اطلاعاتی همچون نرم افزارهای مدیریت ارتباط با مشتریان \LTRfootnote{\label{myfootnote}CRM}  و زیرساخت‌های مناسب به بهبود عملکرد آن‌ها و کسب درآمد و سود بیشتر کمک می‌کند.\\
\item تیم بازاریابی (و به صورت خاص دیجیتال مارکتینگ)\\
اهمیت بازاریابی جهت تولید ارزش برای کسب‌وکار برکسی پوشیده نیست. این امر به خصوص در اکوسیستم امروزی کسب‌وکارها و در مورد کسب‌وکارهای آنلاین اهمیت بیشتری می‌یابد. به طور کلی می‌توان گفت که محتوای پیشرفته می‌تواند مشتریان بالقوه را تا 10٪ افزایش دهد.\\
این تیم مسئولیت ایجاد تجربه کاربری جذاب برای مشتریان را برعهده دارد. وظایف آن‌ها شامل ایجاد محتوای جذاب برای رسانه‌های مختلف، ایجاد کمپین‌های مختلف، تخفیفات و ارائه امکانات خاص به مشتریان، مدیریت باشگاه مشتریان و گزارش درباره فعالیت‌های یکپارچه بازاریابی دیجیتال می‌باشد. در این راستا استفاده از نرم‌افزارها و سامانه‌های اطلاعاتی همچون CRM موثر است.\\
\item تیم حسابداری و مالی\\ 
این تیم مدیریت کلیه جریان‌های مالی و قوانین مرتبط با رگولاتوری و تبعیت از قوانین مالیاتی و … را بر عهده دارند. انجام درست وظایف این تیم، ‌سلامت مالی کسب‌وکار و به تبع آن موفقیت آن را تضمین می‌کند.\\
\item فروشگاه‌هایی که با شرکت قرارداد امضا کرده و همکاری می‌کنند \\

شرکت برای فروش محصولات با فروشگاه‌ها قرارداد می‌بندد و هرچه سود ناشی از فروش محصولات در پلتفرم آنلاین بیشتر باشد، سود فروشگاه‌ها نیز افزایش می‌یابد. به بیان دیگر، بعضی از فروشگاه‌ها فروشی غیر از فروش آنلاین در پلتفرم مذکور ندارند؛ درنتیجه تنها راه کسب درآمد آنها فروش در پلتفرم آنلاین می‌باشد.\\
\item پیک‌های موتوری\\ 
حمل‌و‌نقل روان کالا برای کسب‌وکار‌های آنلاین ضروری است؛ زیرا تحویل به موقع و موثر می‌تواند تجربه مشتری را بهبود دهد که این مسئله به افزایش سود منجر می‌شود. از طرف دیگر، هر چه درآمد شرکت افزایش یابد، درآمد پیک‌های موتوری هم می‌تواند افزایش پیدا کند.\\
\item سرمایه‌گذار\\
حمایت سرمایه‌گذاران از کسب‌وکار به کمک سرمایه‌ای که در اختیار یک کسب‌وکار می‌گذارند،‌ نقش به سزایی در شکل‌گیری اولیه کسب‌وکار و در ادامه برای حیات آن کسب‌وکار دارد. حضور سرمایه‌گذاران خوب می‌تواند مسیر موفقیت کسب‌وکار را به کلی تغییر دهد.\\
\item رقبا\\
شرایط رقابتی در میزان سود و زیان و هم چنین تلاش برای بقای شرکت‌ها بسیار موثر است. رقبا می‌توانند به عنوان تلنگری برای تلاش بیشتر یا اهرم فشار یک کسب‌وکار باشند و نقاط ضعف رقبا می‌توانند به عنوان یک عامل سود‌آور برای کسب‌وکار تلقی شود و برعکس این نیز صادق است. با توجه به این مسئله، رقبا نقش بسیار بزرگی در موفقیت یا عدم موفقیت کسب‌وکار دارند.\\




\subsection{چشم‌اندازهای مختلف ذینفعان و حوزه‌های ارزش آنان}

حوزه‌های ارزش\\

ارزش ذینفعان، ارزش کلی سازمان است که از دیدگاه‌ها و چشم‌اندازهای مختلف دیده می‌شود؛ بنابراین، موضوعی فراتر از ارزش مالی است که از طریق بازده سرمایه‌گذاری اندازه‌گیری می‌شود و ممکن است شامل افزایش عملکرد محصول (چشم‌انداز مشتری)، حاکمیت بهتر (چشم‌انداز حسابرسی)، بهبود شرایط و ضوابط (چشم‌انداز کارمند) یا سیستم‌های پرداخت بهبود‌یافته (تأمین‌کننده) باشد. در واقع، بسیاری از این چشم‌اندازها با هم تعامل دارند و در نهایت، تاثیر ترکیبی از آن‌ها بر عملکرد مالی سازمان، اثر دارد.\\

\item چشم‌انداز مشتری\\
مشتریان از اصلی ترین ذینفعان هر کسب‌وکار و به ویژه کسب‌و‌کار مدنظر می‌باشند. دغدغه این گروه از ذینفعان در رابطه با محصولاتی که خریداری می‌کنند و یا خدماتی که دریافت می‌کنند، مرتبط با قیمت، کیفیت، در دسترس بودن، تحویل و خدمات پس از فروش می‌باشد. همچنین شیوه برخورد کسب‌وکار با آن‌ها از اهمیت زیادی برخوردار است.یکی از راه‌های افزایش ارزش ذینفعان از دیدگاه مشتری تمرکز بر بازاریابی رابطه‌ای است. این رویکرد با استفاده از تحقیقات بازار، نیازها، خواسته‌ها، علایق و ارزش‌های مشتریان واقعی و بالقوه را تعیین می‌کند.\\
\item ‌چشم‌انداز تأمین‌کنندگان\\
ارزش برای تامین‌کنندگان از طریق دستیابی به قیمت مناسب و بازار منظم کالاها و خدمات فراهم می‌شود.\\
روابط تأمین‌کننده می‌تواند از طریق پشتیبانی از فعالیت‌های تأمین‌کننده توسعه یابد. برخی از سازمان‌ها تأمین‌کنندگان خود را به طور فعال درگیر فرآیندهای ارتباطی می‌کنند (به عنوان مثال در مورد راه‌هایی که تأمین‌کننده به محصولات و خدمات سازمان کمک می‌کند) و برخی حتی بیشتر پیش می‌روند، به تامین‌کنندگان خود مشاوره داده و کمک می‌کنند تا با مشکلات مدیریتی و عملیاتی خود مقابله کنند. عملکرد تأمین‌کنندگان با معیارهایی همچون سرعت تحویل یا سرعت پاسخ به الزامات غیر معمول سنجیده می‌شود. در کسب‌وکار مورد بحث تامین‌کنندگان همان فروشگاه‌های طرف قرارداد هستند که فروشگاه آنلاین را تشکیل می‌دهند.\\
 \item چشم‌انداز کارکنان\\
ارزش به کارکنان از طریق حقوق، دستمزد و مزایا، امنیت شغلی، رفتار کارفرما با آن‌ها (سبک مدیریتی)، فرصت‌های آموزش، توسعه و ارتقا، ، بهداشت و ایمنی پیرامون کار فراهم می‌شود.از اقدامات مهم درحوزه ارزش رساندن به کارکنان توجه مناسب به امر مدیریت سرمایه انسانی است. کارکنان کسب‌وکار مورد بحث، شامل تمامی تیم‌های عملیاتی همچون IT، فروش و بازاریابی، حسابداری و مالی می‌باشد.\\
 \item چشم‌انداز حسابرسی و سرمایه\\
حسابرسان و سرمایه‌گذاران با مدیریت بازرگانی و مالی سازمان در ارتباط هستند و ارزش به آنان از آن حوزه تامین می‌شود. بازتاب این امر در ساختار امور مالی از نظر مدیریت دارایی، جریان‌های نقدی و بازده سرمایه‌گذاری و ساختار حاکمیت شرکتی دیده می‌شود. حاکمیت شرکت به نحوه اداره کسب‌وکار توسط مدیریت ارشد و به‌ویژه نحوه انتخاب تیم مدیریت عالی، نحوه پاداش مدیران عالی و نحوه کنترل و حسابرسی امور مالی سازمان مربوط می‌شود.\\
\item ‌چشم‌انداز دولت\\
دولت به توانایی سازمان‌ها برای کمک به مالیات عمومی، اشتغال و رفاه جامعه علاقه مند است. دولت همچنین موظف است اطمینان حاصل کند که سازمان‌ها به درستی اداره می‌شوند، در فعالیت‌هایشان اخلاقی و مطابق با قوانین تنظیم‌کننده کسب‌وکار، اشتغال، رقابت، بهداشت و ایمنی و تهیه کالاها و خدمات عمل می‌کنند. \\




\subsection{ویژگی‌های اصلی و منحصر به فروش فروشگاه آنلاین}

\subsubsection{ویژگی‌های اصلی و الزامی فروشگاه‌های آنلاین}


\end{itemize}

\begin{enumerate}

\item منوی پیمایشی با دسته‌بندی‌های مشخص و دقیق جهت جست وجوی محصولات مختلف
\newline
وجود منوی پیمایشی در هر فروشگاه انلاین ضروری به‌نظر می‌رسد. آنچه اهمیت دارد، آن است که در ارتباط با منوی پیمایشی دسته‌بندی‌های دقیق و شفاف محصولات براساس معیارهای صحیح تاثیر به سزایی در جذب مشتریان و تشویق آنان به ادامه خرید دارد. به طورکلی ساختار وبسایت علاوه بر این که از لحاظ فنی باید از کیفیت مناسبی برخوردار باشد، باید محیط ساده و کاربرپسندی نیز داشته باشد تا عموم کاربران در استفاده از آن مشکلی نداشته باشند. 
\item توضیحات دقیق در مورد هر محصول
\newline
ذکر توضیحات مفصل و با جزئیات کالا در ارتباط با تمامی محصولات یک فروشگاه آنلاین ضروری است. این امر در مورد کالاهای مشابه که تفاوت اندکی دارند،‌بسیار اهمیت دارد.(درارتباط با فروشگاه مورد بررسی ذکر جزئیاتی همچون جنس محصول، طرح،سایز، نام تولیدکننده،مورد استفاده محصول و…)به طور کلی ویژگی‌های هر محصولی باید به طور دقیق ذکر شود اما به گونه ای که کاربر در کمترین زمان بیشترین اطلاعات را به دست آورد.
\item تصاویر باکیفیت و جزئیات از محصولات
\newline
ارائه تصاویر باکیفیت و از زوایای مختلف از محصول (در مورد پوشاک:تا حد امکان ارائه تصویری از تن‌خور آن محصول) به همراه قابلیت بزرگنمایی اهمیت بسیاری دارد.
\item ویدیوهای معرفی محصولات و ویژگی‌های آنان
\newline
ارائه جزئیات دقیق از محصول به صورت سه‌بعدی در قالب ویدیو در آگاهی کاربران از محصول انتخابی اهمیت زیادی داشته و کاربران را در وضعیت تصور استفاده از محصول قرار داده و آن‌ها را به خرید آن محصول ترغیب می‌کند.
\item بخش نظرات کاربران
\newline
وجود این بخش در وبسایت هم برای کاربران و هم کسب‌وکار اهمیت دارد. مزیت وجود این بخش برای کاربران در آگاهی از نظرات کاربرانی است که تجربه استفاده از محصول را داشته‌اند و به این ترتیب کاربران اعتماد بیشتری درمورد انتخاب یا عدم انتخاب آن محصول پیدا میکنند. برای کسب‌وکار نیز نظرات منفی میتوانند بسیار سودمند باشند؛‌ هرچند که ممکن است وجود این بخش در وبسایت به لحاظ نظرات منفی کاربران با ریسک همراه باشد اما به هر شکل اطلاعات مفیدی در ارتباط با قطع همکاری با تامین‌کنندگانی که کالای آن‌ها نارضایتی بیشتری در پی دارد، فراهم میکند و امکان کار بر روی نقاط ضعف را فراهم می‌کند.
\item صفحه جذاب و حاوی اطلاعات«درباره ما» با جزئیات تماس
\newline
ذکر اطلاعات کلی کسب‌وکار همچون تاریخچه مختصر، صاحبان کسب‌وکار، مکان دفتر مرکزی و راه‌های ارتباطی در صفحه درباره ما سبب جلب اعتماد کاربران شده و به بازدیدکنندگان اطمینان می‌بخشد که با افراد حقیقی در واقع در تعاملند. به‌علاوه به برند کسب‌وکار نیز چهره و شخصیت می‌بخشد.
\item بخش سوالات متداول و پاسخ به آنان
\newline
وجود این بخش نه تنها باعث می‌شود مشتریان بهتر آگاه شوند بلکه سبب صرفه‌جویی در وقت تیم فروش کسب‌وکار نیز می‌شود. در ارتباط با این بخش باید اطمینان حاصل شود که پاسخ‌ها به وضوح و مختصر نوشته شده‌اند تا کاربران پاسخ سوالات خود را به‌راحتی بیابند. نمونه‌ای از این سوالات در کسب‌وکار مورد بررسی عبارتنداز: چه زمانی طول می‌کشد تا محصول ارسال شود؟ هزینه ارسال محصول چقدر است؟ راه‌های مختلف پرداخت کدامند؟ روال مرجوعی کالا به چه صورت است؟ و...
\item عملکرد خوب و مناسب بر روی پلتفرم تلفن همراه
\newline
بخش عمده کاربران کسب‌وکارهای آنلاین امروزه از طریق تلفن همراه با آن‌ها ارتباط برقرار می‌کنند؛ بنابراین عملکرد مناسب وبسایت و در دسترس بودن تمام قابلیت‌های آن بر روی تلفن همراه درست مانند کار با وبسایت بر روی دسکتاپ اهمیت دارد و در صورت عدم توجه به این امر بخش عمده‌ای از کاربران نادیده گرفته می‌شوند.
\item امکان پرداخت از طرق مخلتف
\newline
کاربران مختلف روش‌های پرداخت مختلفی را ترجیح می‌دهند؛ برخی روش‌های سنتی و درمحل را ترجیح می‌دهند، درحالی که بسیاری روش‌های آنلاین پرداخت را انتخاب می‌کنند. توجه به دسته‌های مختلف کاربران و فراهم کردن روش‌های پرداخت متفاوت و درگاه‌های مختلف پرداخت،‌ اهمیت دارد.
\item امکان ردیابی سفارشات در طی مراحل مختلف از آغاز سفارش تا تحویل
\newline
امکان ردیابی، مشتریان را قادر می‌سازد زمان تحویل محصول را پیش‌بینی کنند، این امر خصوصاً در مورد پرداخت وجه نقد هنگام تحویل مناسب است. همچنین آگاهی از مراحل مختلفی که محصول طی می‌کند، در ایجاد حس امنیت در مشتری تاثیرگذار است.
\item امکان گفتوگوی زنده با تیم فروش هنگام بروز مشکل یا ایجاد سوال
\newline
فراهم کردن چنین امکانی برای رفع سوالات و مشکلات و چالش‌های کاربران حین مسیر خرید و همچنین قبل و بعد از آن اهمیت زیادی دارد؛ اما نکته حائز اهمیت آن است که این گفت‌وگو و پاسخ و خدمت‌رسانی به مشتری باید با سرعت بسیار خوبی انجام شود و درغیر اینصورت ارزش افزوده مدنظر را ایجاد نمی‌کند.
\item حضور فروشگاه در شبکه‌های اجتماعی و فعالیت در آن‌ها
\newline
امروزه بخش مهمی از مشتریان کسب‌وکارهای آنلاین از طریق شبکه‌های اجتماعی فراهم می‌شوند و علاقه‌مندند که از طریق پلتفرم‌های مختلفی با کسب‌وکار در ارتباط باشند؛‌ لذا حضور در شبکه‌های اجتماعی برای برقراری ارتباط موثر با مشتریان فعلی و نیز برگزاری انواع کمپین تبلیغاتی و … برای جذب مشتریان جدید اهمیت دارد.
\item امکان تشکیل لیست علاقه مندی‌ها 
\newline
لیست علاقه مندی‌ها به مشتریان این امکان را میدهد تا تمام محصولاتی را که دوست دارند در یک مکان ذخیره کنند. بعضی از آن‌ها ممکن است در آینده توسط خود مشتریان خریداری شوند. فراهم کردن این امکان و به کارگیری روش‌های یادگیری ماشین و داده کاوی این لیست‌ها می‌تواند سبب به‌دست آوردن دید بهتر از مشتری و ارائه پیشنهادات موثری به وی شود که می‌تواند برای کسب‌وکار ارزش ایجاد کند.\\


\subsubsection{ویژگی‌های منحصر به فروش فروشگاه آنلاین مورد بحث}


\item فروش در شهر تهران
\newline
موقعیت فیزیکی یک مزیت بزرگ برای کسب‌وکار محسوب می‌شود. در شهر تهران جمعیت 5.17 درصد جمعیت کل کشور را پوشش می‌دهد که حائز اهمیت است و سبب ایجاد بازار محلی گسترده‌ای می‌شود و به‌علت جمعیت بیشتر تعداد نیروی کار نیز زیاد است (به خصوص که از شهرهای اطراف تهران نیز برای کار به تهران می‌آیند). همچنین دسترسی به اینترنت و گوشی‌های هوشمند و پلتفرم‌های ارتباطی در این شهر بیشتر از سایر شهرهاست که سبب افزایش مشتری بالقوه بیشتر می‌شود. 

\item قیمت متفاوت و به‌صرفه پوشاک برند
\newline
رقابت به واسطه قیمت‌گذاری مناسب پوشاک برند میتواند یک مزیت بزرگ فروشگاه آنلاین است. در کسب‌وکار آنلاین مورد بحث، پوشاک برند به صورت انبوه و بدون واسطه خریداری می‌شود. به این‌صورت که در هر دوره، افرادی به‌عنوان خریدار به محل‌های مختلف (داخل کشور یا خارج) فرستاده می‌شوند و به طورمستقیم از تولیدی یا فروشگاه‌های برندهای معتبر خرید انجام می‌دهند که این سیاست در قیمت گذاری مناسب اجناس نقش بسزایی دارد.

\item فروش اجناس خاص
\newline
به عنوان مثال، فروشگاه آنلاین مورد بحث قرارداد‌های مختلفی با برند‌های ایرانی در حوزه صنعت بافندگی و یا چرم دارد و این برند‌ها انواع لباس‌های بافتنی (دستباف) و چرم (طبیعی یا مصنوعی مرغوب) را تولید می‌کنند که بسیار خاص و کمیاب می‌باشند.
\item قابلیت‌های ویژه برای اطمینان از کیفیت و طراحی لباس‌ها
\newline
یکی از بزرگترین دغدغه‌های خرید لباس این است که بعد از شست‌وشو، کیفیت لباس تغییری نکند. فروشگاه مورد بحث، امکان مرجوعی بعد از شست‌وشو را فراهم می‌کند که اگر با یک بار شست‌و‌شوی لباس خریداری شده، کیفیت لباس افت کرد، مشتری می‌تواند لباس خریداری شده را بازگرداند. این اقدام نشان می‌‌دهد که فروشگاه به کیفیت محصولات خود اطمینان دارد و این اطمینان را به مشتریان خود نیز انتقال می‌دهد.
دغدغه دیگر مشتریان هنگام خرید لباس این است که در وهله اول عیب‌های طراحی لباس مشاهده نمی‌شود. در نتیجه فیلم‌هایی در محل‌ها و موقعیت‌های آب وهوایی مختلف توسط فردی که لباس را پوشیده است، گرفته می‌شود که به تصمیم‌گیری خرید مشتری کمک کند.

\item قابلیت ویژه برای مشتریان وفادار
\newline
فروشگاه مورد بحث، برای مشتریان با رتبه طلایی در باشگاه مشتریان، امکان اتوشویی رایگان لباس‌های مجلسی و خاص را فراهم می‌کند که مشتریان می‌توانند لباس خود درب منزل تحویل دهند و تحویل بگیرند.(البته فروشگاه خود این عمل را انجام نمی‌دهد؛ بلکه این کار به صورت برون‌سپاری و با همکاری با کسب‌وکارهای با درصد اطمینان بالا در این حوزه انجام می‌شود)

\item فروش به صورت آنلاین از طریق پلتفرم‌های مختلف
\newline
مشتریان دوست دارند به روش‌های مختلف خرید کنند. Omnichannel به معنای دیدار با مشتریان خود در جایی است كه آن‌ها هستند. حضور در پلتفرم‌های مختلف از جمله وبسایت، صفحه اینستاگرام، کانال تلگرامی و اپلیکیشن، سبب می‌شود که راه دسترسی مشتری به فروشگاه ما آسان شود.

\item استفاده از نرم افزار CRM 
\newline
این نرم افزار به کسب‌وکار کمک می‌کند تا مدیریت کانال‌های مختلف در ارتباط با مشتری به خوبی انجام شود. پاسخگویی به موقع به مشتریان یکی از عوامل موثر در فروش و رضایت مشتری است که نرم افزار CRM در این امر به تیم فروش کمک می‌نماید.

\item پرو کردن قبل از خرید
\newline
در خرید آنلاین برخلاف خرید حضوری این امکان فراهم نیست که قبل از خرید، لباس پوشیده شود و در خرید لباس اندازه، رنگ، طرح بسیار اهمیت دارد. در فروشگاه آنلاین مورد بحث این قابلیت فراهم شده‌است که مشتریان با رتبه طلایی و نقره ای می‌توانند ( با دریافت کارت معتبری به عنوان اطمینان برای پس دادن لباس‌ها) 5 لباس را انتخاب کنند و قبل از خرید، لباس را امتحان کنند.


\item گزینه اهدا به امور خیریه
\newline
فروشگاه مورد بحث با موسسات امور خیریه (بیماری‌های خاص مانند: سرطان، اوتیسم و …) قرار می‌بندد و در مناسبت‌های خاص مانند روز جهانی اوتیسم، طرح خیریه‌ای را در پلتفرم خود ایجاد می‌کند که مشتریان می‌توانند لباسی را برای امور خیریه خریداری کنند که فروشگاه مبلغ پرداختی و لباس خریداری شده را به عنوان اهدایی آن فرد به موسسات امور خیریه اهدا می‌کند.
\end{enumerate}
\section{داستان کاربری}


\begin{itemize}
\item سناریو: کاربر ثبت‌نام نشده فرم ثبت‌نام را مشاهده می‌کند.
\newline
به عنوان یک کاربر ثبت‌نام نشده
\newline
وقتی می‌خواهم روی "ثبت‌نام" کلیک کنم
\newline
باید فرمی را مشاهده کنم که به من امکان ثبت‌نام را بدهد.  

\item سناریو: کاربر ثبت‌نام نشده پیامک تایید یا ایمیل احراز هویت را دریافت می‌کند.
\newline
به عنوان یک کاربر ثبت‌نام نشده
\newline
وقتی می‌خواهم اطلاعات شماره همراه یا ایمیل را برای ثبت‌نام ثبت کنم
\newline
باید پیامک رمز تایید به شماره‌همراه یا ایمیل احراز هویت فرستاده شود.

\item سناریو :‌ فرایند ثبت‌نام کاربر ثبت‌نام نشده تکمیل نشود.
\newline
به عنوان یک کاربر ثبت‌نام نشده
\newline
وقتی می‌خواهم اطلاعات شماره همراه یا ایمیل را برای ثبت‌نام ثبت کنم، در صورت رخداد خطا در هر کدام از مراحل فرایند
\newline
باید پیغامی مبنی بر رخداد خطا و عدم تکمیل ثبت‌نام مشاهده کنم و به صفحه ثبت اطلاعات بازگشت‌داده شوم.
\item سناریو: کاربر ثبت‌نام‌شده که وارد سایت شده به دلیل فراموشی رمز عبور نتواند وارد پروفایل کاربری خود شود.
\newline
به عنوان کاربر ثبت‌نام‌شده
\newline
وقتی می‌خواهم وارد سایت شوم و رمز عبور را فراموش کرده‌ام
\newline
امکان تغییر رمز و دریافت رمز جدید مهیا باشد.

\item سناریو: کاربر ثبت‌نام‌شده لیست فروشگاه‌های لباس نزدیک به وی و هزینه پیک را مشاهده می‌کند.
\newline
به عنوان یک کاربر ثبت‌نام‌شده
\newline
وقتی می‌خواهم سفارشی را ایجاد کنم
\newline
باید لیست فروشگاه‌های لباس نزدیک به خود و هزینه پیک را مشاهده کنم.

\item سناریو: کاربر ثبت‌نام‌شده لیست کالای فروشگاه مورد نظر را مشاهده می‌کند.
\newline
به عنوان کاربر ثبت‌نام‌شده
\newline
وقتی فروشگاه مدنظر را انتخاب می‌کنم
\newline
باید لیست کالاهای موجود در آن فروشگاه را مشاهده کنم.

\item سناریو: کاربر ثبت‌نام‌شده پس از مشاهده لیست کالای فروشگاه موردنظر،محصولات مدنظر خود را به سبد خرید اضافه کند.
\newline
به عنوان کاربر ثبت‌نام‌شده
\newline
وقتی محصولی را به تعداد مدنظر انتخاب می‌کنم
\newline
باید نام و تعداد محصولات به سبد خرید اضافه شود.


\item سناریو: کاربر ثبت‌نام‌شده تایید سیستم را به منظور تایید لیست کالاهای مورد انتخابی دریافت می‌کند.
\newline
به عنوان کاربر ثبت‌نام‌شده
\newline
وقتی لیست کالاهای مدنظر جهت سفارش را انتخاب می‌کنم
\newline
باید پیامی از سوی سیستم به منظور تایید لیست کالاهای سفارش، دریافت کنم. 


\item سناریو: کاربر ثبت‌نام‌‌شده اخطار سیستم را به منظور رد لیست کالاهای انتخابی دریافت می‌کند.
\newline
به عنوان کاربر ثبت‌نام‌شده
\newline
وقتی لیست کالاهای  مدنظر جهت سفارش را انتخاب می‌کنم
\newline
باید اخطار سیستم به منظور رد لیست کالاهای سفارش به همراه علت آن را دریافت کنم. 


\item سناریو: کاربری که قصد نهایی کردن سفارش خود را دارد، به صفحه درگاه بانکی انتقال داده می‌شود.
\newline
به عنوان کاربری که قصد نهایی کردن سفارش خود را دارد
\newline
وقتی می‌خواهم هزینه سفارش خود را واریز نمایم وسفارش را ثبت کنم
\newline
باید به صفحه درگاه بانکی انتقال داده شوم.

\item سناریو : به عنوان کاربری که اطلاعات کارت بانکی خود وارد کرده‌است، پیامک رمز پویا از سیستم بانکی را دریافت کند.
\newline
به عنوان کاربری که اطلاعات کارت بانکی خود وارد کرده‌است
\newline
وقتی اطلاعات کارت بانکی خود را وارد کنم 
\newline
باید پیامک رمز پویا را از سیستم بانکی دریافت کنم.

\item سناریو : در صورت ورود نادرست اطلاعات کارت بانکی یا رمز پویا و یا عدم وجود موجودی کافی در حساب شخص به کاربری که در مرحله ثبت نهایی سفارش است، اخطار داده شود.
\newline
به عنوان کاربری که در مرحله ثبت نهایی سفارش در درگاه انتقال پرداخت است
\newline
وقتی اطلاعات کارت بانکی نادرست وارد‌شده و یا موجودی کافی برای خرید ندارم
\newline
باید اخطاری به همراه علت مرتبط از سوی سیستم دریافت کنم.

\item سناریو : به عنوان کاربری که واریز خود را انجام داده‌است، تایید موفقیت پرداخت را از سیستم دریافت می‌کند.
\newline
به عنوان کاربری که واریز خود را انجام داده‌است
\newline
وقتی پرداخت مبلغ سفارش خود را تکمیل میکنم
\newline
باید تایید موفقیت پرداخت از سیستم را دریافت کنم

\item سناریو : کاربری که پرداخت آن ناموفق بوده‌است، به لیست خرید خود بازگشت‌داده شود.
\newline
به عنوان کاربری که پرداخت آن ناموفق بوده‌است
\newline
وقتی پرداخت مبلغ سفارش خود ناموفق است
\newline
باید به لیست خرید خود بازگشت‌داده شوم و امکان مشاهده مجدد لیست و اضافه یا کم کردن محصولات از لیست مهیا باشد.

\item سناریو : کاربری که سفارش خود را دریافت کرده‌است، امکان به اشتراک گذاشتن نظرات خود را داشته باشد.
\newline
به عنوان کاربری که سفارش خود را دریافت کرده‌است
\newline
وقتی سفارش را دریافت می‌کنم
\newline
باید فرم ثبت نظر و امتیاز را مشاهده کنم.

\item سناریو : کاربری که قصد مرجوعی کالای خریداری‌شده را دارد، با تیم پشتیبانی فروش ارتباط برقرار می‌کند.
\newline
به عنوان کاربری که قصد مرجوعی کالای خریداری‌شده را دارد
\newline
وقتی قصد مرجوعی کالای خریداری‌شده را دارم
\newline
باید امکان تماس با تیم پشتیبانی فروش را داشته‌باشم.

\item سناریو : کاربری که با قصد مرجوعی کالای خریداری‌شده با تیم پشتیبانی فروش تماس می‌گیرد، تایید تیم پشتیبانی فروش را دریافت می‌کند.
\newline
به عنوان کاربری که با قصد مرجوعی کالای خریداری‌شده با تیم پشتیبانی فروش تماس می‌گیرد
\newline
وقتی با تیم پشتیبانی فروش تماس می‌گیرم، در صورت مهیا بودن شرایط
\newline
باید تایید تیم پشتیبانی فروش را دریافت کنم.

\item سناریو : کاربری که با واحد پشتیبانی هماهنگی درخصوص مرجوع کردن محصول را انجام داده، محصول را به پیک موتوری تحویل می‌دهد.
\newline
به عنوان کاربری که قصد برگشت دادن کالای خود را دارد
\newline
وقتی میخواهم کالای خود را برگشت دهم
\newline
باید آن را به پیک موتوری تحویل دهم.

\item سناریو : کاربری که کالای خود را بازگردانده است، مبلغ کالای بازگشتی به او بازگردانده شود.
\newline
به عنوان کاربری که کالای خود را بازگردانده‌است
\newline
وقتی کالای خود برگشت می‌دهم
\newline
باید مبلغ کالای بازگشتی را دریافت کنم.


\section{نیازمندی‌های کاربردی }
نیازمندی‌های کاربردی را به‌طور کلی به سه دسته میتوان تقسیم نمود:\\
-الزامات مدیریت: شامل اقداماتی که کارمندان و کارکنان توسط سیستم انجام می‌دهند. \\
-الزامات بازاریابی : انواع فعالیت‌های بازاریابی که وبسایت از آن‌ها پشتیبانی می‌کند. \\
-الزامات مرتبط با فروش : اقداماتی که تیم مدیریت و فروش باید انجام دهد.\\
برخی از مهم‌ترین نیازمندی‌های کاربردی از دسته‌های مختلف به همراه توضیحات هریک در ادامه آورده می‌شود:\\
دسترسی به پنل مدیریت: از بین کارمندان با سطوح دسترسی معین، برخی بایستی به پنل مدیریت دسترسی داشته و بتوانند تغییرات لازم را از آن طریق اعمال کنند و موارد نیازمند پشتیبانی را پیگیری نمایند.\\
بررسی سابقه سفارش : کارمندان بخش‌های مختلف به‌خصوص فروش و مدیران این بخش نیاز دارند تا بتوانند سابقه فروش و ارقام و سایر جزئیات مرتبط با آن را در قالب داشبوردهای مدیریتی و اطلاعات با قالب‌های مختلف مشاهده کنند و در تصمیم گیری درخصوص سیاست‌ها و نیز کارمندان خود با اطلاعات کافی عمل کنند.\\
ایجاد فاکتور: پس از ثبت سفارش موفق کاربران (مشتریان) لازم است نسخه الکترونیکی فاکتور خرید به وی نمایش داده‌شود تا در صورت لزوم تغییرات لازم را پیش از پرداخت نهایی در آن ایجاد کند.\\
تغییر قیمت ها : افراد مرتبط با قیمت‌گذاری شامل مدیران سطح بالا و برخی مدیران میانی نیازمند ساختاری جهت تغییر قیمت‌ها در صورت لزوم هستند.\\
مدیریت طبقه‌بندی کالاها : همانطور که پیش‌تر ذکر شد، طبقه‌بندی کالاها تاثیر فوق العاده‌ای در انتقال یک تجربه خوب کاربری و حفظ کاربر در سایت دارد. به این منظور باید افراد مرتبط، امکان تغییر و به روزرسانی طبقه‌بندی کالاهای مختلف در سایت را داشته‌باشند.\\
ایجاد تخفیف و کد تبلیغاتی : برای افراد بخش بازاریابی در اختیار داشتن امکاناتی  برای ایجاد کد تخفیف و کارت هدیه با توجه به مناسبت‌های مختلف و برمبنای مشتریان و اطلاعات تاریخچه‌ای آن‌ها که در نرم‌افزارهای CRM و باشگاه مشتریان قرار دارد، اهمیت زیادی دارد.\\
تغییر قوانین قیمت سبد خرید: وجود امکانات خودکارسازی جریان کار جهت اعمال تخفیفات بر روی سبد خرید برمبنای قوانین مشخص یک نیازمندی مهم دیگر است.\\
 به حداقل رساندن مراحل خرید: این مورد بالاترین اولویت را در بین نیازهای فروش داراست و  به‌ویژه باید اطمینان کسب نمود که روند پرداخت در سریع‌ترین زمان ممکن انجام می‌شود.علاوه‌بر این ، نماد سبد خرید باید به وضوح در هر صفحه از سایت قابل مشاهده باشد. این کار تعامل خریدار بالقوه را برای کار با سبد خرید آسان‌تر می‌کند (مشاهده کالاهای انتخاب شده، حذف موارد غیر ضروری، تنظیم مقدار واحد محصول موردنیاز و ...)\\
عملکرد مناسب در پلتفرم‌های تلفن همراه و تبلت: امروزه ، تعداد بسیار زیادی از کاربران اینترنت خریدهای آنلاین را از طریق تلفن‌های هوشمند و تبلت‌ها انجام می‌دهند تا از طریق لپ‌تاپ و رایانه‌های شخصی؛ لذا این مورد یکی از الزامات کاربردی هر کسب‌وکار آنلاین می‌باشد.\\
طٰراحی منحصر به‌فرد و متمایزکننده: ویژگی دیگری که یک وبسایت یک کسب‌وکار آنلاین را تعریف می‌کند ، طراحی منحصر به‌فرد و معتبر آن است.نیاز به شخصی‌سازی وبسایت یک مولفه مهم در طراحی آن است.\\
مطالب مرتبط و مفید: قراردادن محتوا و مقالات تخصصی مرتبط با کسب‌وکار و محصولات علاوه بر مزیت بهبود SEO سایت، راهی برای تأمین نیازهای دقیق ترین مشتریانی است که به دنبال چیزی بیش از توصیف استاندارد محصول هستند.\\
ابزارهای خبرنامه ایمیل:خبرنامه از جمله موارد ضروری بازاریابی است که لازمه وبسایت هر کسب‌وکار آنلاین است و  این امکان را فراهم می‌کند تا بیشتر مراحل تعامل فعال با مشتریان فروشگاه آنلاین از طریق ایمیل خودکار شود.\\ 
یکپارچگی با شبکه‌های اجتماعی: امکان اتصال به  شبکه‌های اجتماعی محبوب از طریق API و اختصاص بخشی از وبسایت برای رتبه‌بندی و افرودن نظرات کاربران در آن دو نیازمندی مهم هستند که باید مدنظر قرار گیرند. این  موضوع میزان اعتماد کاربر به سیستم تجارت را کسب وکار را افزایش می‌دهد و نیاز وی را برای ارزیابی محصول برآورده می‌کند.\\
یکپارچه‌سازی سیستم‌های حمل‌ونقل و پرداخت: این موضوع که کاربران برای پرداخت‌ها و نحوه ارسال کالای انتخابی(سیستم حمل‌ونقل) بتوانند از میان چند گزینه ترجیحات خود را برگزینند، اهمیت زیادی دارد و لذا سیستم باید با سیستم‌های حمل‌ونقل و پرداخت متفاوتی یکپارچه شود.\\
گفت‌وگوی برخط: گفت‌وگوی برخط نکته‌ای است که باید در میان الزامات اساسی مدیریت مورد توجه قرار گیرد. هر چقدر محصول ساده باشد ، داشتن این امکان در سایت که مشتریان می‌توانند توسط آن بلافاصله پاسخ سوالات خود را دریافت کنند، بسیار مطلوب است و توصیه می‌شود برخی از الگوریتم‌های ساده را برای یک بات \LTRfootnote{\label{myfootnote}bot}گفت‌وگو شبیه‌سازی کمد که این امر باعث کاهش تعداد تماس‌ها با اپراتورهای زنده می شود.\\
مجموعه‌ای مفید از فیلترها:هرچه امکانات جست‌وجوی پیشرفته بیشتری برای وبسایت فراهم شود، سبب بهتر یافتن محصولات توسط کاربران و خرید بیشتر و در درازمدت سود بیشتر کسب‌وکار می‌شود.
\end{itemize}
\end{flushright}
\end{document}
