
\documentclass[14pt]{article}
\usepackage{xepersian} 

\settextfont{XB Zar}
\begin{document} 
\section{داستان کاربری}

\begin{flushright}
\begin{itemize}
\item سناریو: کاربر ثبت نام نشده فرم ثبت نام را مشاهده می کند.
\newline
به عنوان یک کاربر ثبت نام نشده
\newline
وقتی می خواهم روی "ثبت نام" کلیک کنم
\newline
باید فرمی را مشاهده کنم که به من امکان ثبت نام را بدهد.  

\item سناریو: کاربر ثبت نام نشده پیامک تایید یا ایمیل احراز هویت را دریافت می کند.
\newline
به عنوان یک کاربر ثبت نام نشده
\newline
وقتی می خواهم اطلاعات شماره همراه یا ایمیل را برای ثبت نام ثبت کنم
\newline
باید پیامک رمز تایید به شماره همراه یا ایمیل احراز هویت فرستاده شود.

\item سناریو :‌ فرایند ثبت نام کاربر ثبت نام نشده تکمیل نشود.
\newline
به عنوان یک کاربر ثبت نام نشده
\newline
وقتی می خواهم اطلاعات شماره همراه یا ایمیل را برای ثبت نام ثبت کنم، در صورت رخداد خطا در هر کدام از مراحل فرایند
\newline
باید پیغامی مبنی بر رخداد خطا و عدم تکمیل ثبت نام مشاهده کنم و به صفحه ثبت اطلاعات بازگشت داده شوم.
\item سناریو: کاربر ثبت نام شده که وارد سایت شده به دلیل فراموشی رمز عبور نتواند وارد پروفایل کاربری خود شود.
\newline
به عنوان کاربر ثبت نام شده
\newline
وقتی می خواهم وارد سایت شوم و رمز عبور را فراموش کرده ام
\newline
امکان تغییر رمز و دریافت رمز جدید مهیا باشد.

\item سناریو: کاربر ثبت نام شده لیست فروشگاه های لباس نزدیک به وی و هزینه پیک را مشاهده می کند.
\newline
به عنوان یک کاربر ثبت نام شده
\newline
وقتی می خواهم سفارشی را ایجاد کنم
\newline
باید لیست فروشگاه های لباس نزدیک به خود و هزینه پیک را مشاهده کنم.

\item سناریو: کاربر ثبت نام شده لیست کالای فروشگاه مورد نظر را مشاهده میکند.
\newline
به عنوان کاربر ثبت نام شده
\newline
وقتی فروشگاه مدنظر را انتخاب میکنم
\newline
باید لیست کالاهای موجود در آن فروشگاه را مشاهده کنم.

\item سناریو: کاربر ثبت نام شده پس از مشاهده لیست کالای فروشگاه موردنظر،محصولات مدنظر خود را به سبد خرید اضافه کند.
\newline
به عنوان کاربر ثبت نام شده
\newline
وقتی محصولی را به تعداد مدنظر انتخاب می کنم
\newline
باید نام و تعداد محصولات به سبد خرید اضافه شود.


\item سناریو: کاربر ثبت نام شده تایید سیستم را به منظور تایید لیست کالاهای مورد انتخابی دریافت میکند.
\newline
به عنوان کاربر ثبت نام شده
\newline
وقتی لیست کالاهای مدنظر جهت سفارش را انتخاب میکنم
\newline
باید پیامی از سوی سیستم به منظور تایید لیست کالاهای سفارش، دریافت کنم. 


\item سناریو: کاربر ثبت نام شده اخطار سیستم را به منظور رد لیست کالاهای انتخابی دریافت میکند.
\newline
به عنوان کاربر ثبت نام شده
\newline
وقتی لیست کالاهای  مدنظر جهت سفارش را انتخاب میکنم
\newline
باید اخطار سیستم به منظور رد لیست کالاهای سفارش به همراه علت آن را دریافت کنم. 


\item سناریو: کاربری که قصد نهایی کردن سفارش خود را دارد، به صفحه درگاه بانکی انتقال داده می شود.
\newline
به عنوان کاربری که قصد نهایی کردن سفارش خود را دارد
\newline
وقتی میخواهم هزینه سفارش خود را واریز نمایم وسفارش را ثبت کنم
\newline
باید به صفحه درگاه بانکی انتقال داده شوم.

\item سناریو : به عنوان کاربری که اطلاعات کارت بانکی خود وارد کرده است، پیامک رمز پویا از سیستم بانکی را دریافت کند.
\newline
به عنوان کاربری که اطلاعات کارت بانکی خود وارد کرده است
\newline
وقتی اطلاعات کارت بانکی خود را وارد کنم 
\newline
باید پیامک رمز پویا را از سیستم بانکی دریافت کنم.

\item سناریو : در صورت ورود نادرست اطلاعات کارت بانکی یا رمز پویا و یا عدم وجود موجودی کافی در حساب شخص به کاربری که در مرحله ثبت نهایی سفارش است، اخطار داده شود.
\newline
به عنوان کاربری که در مرحله ثبت نهایی سفارش در درگاه انتقال پرداخت است
\newline
وقتی اطلاعات کارت بانکی نادرست وارد شده و یا موجودی کافی برای خرید ندارم
\newline
باید اخطاری به همراه علت مرتبط از سوی سیستم دریافت کنم.

\item سناریو : به عنوان کاربری که واریز خود را انجام داده است، تایید موفقیت پرداخت را از سیستم دریافت می کند.
\newline
به عنوان کاربری که واریز خود را انجام داده است
\newline
وقتی پرداخت مبلغ سفارش خود را تکمیل میکنم
\newline
باید تایید موفقیت پرداخت از سیستم را دریافت کنم

\item سناریو : کاربری که پرداخت آن ناموفق بوده است، به لیست خرید خود بازگشت داده شود.
\newline
به عنوان کاربری که پرداخت آن ناموفق بوده است
\newline
وقتی پرداخت مبلغ سفارش خود ناموفق است
\newline
باید به لیست خرید خود بازگشت داده شوم و امکان مشاهده مجدد لیست و اضافه یا کم کردن محصولات از لیست مهیا باشد.

\item سناریو : کاربری که سفارش خود را دریافت کرده است، امکان به اشتراک گذاشتن نظرات خود را داشته باشد.
\newline
به عنوان کاربری که سفارش خود را دریافت کرده است
\newline
وقتی سفارش را دریافت می کنم
\newline
باید فرم ثبت نظر و امتیاز را مشاهده کنم.

\item سناریو : کاربری که قصد مرجوعی کالای خریداری شده را دارد، با تیم پشتیبانی فروش ارتباط برقرار می کند.
\newline
به عنوان کاربری که قصد مرجوعی کالای خریداری شده را دارد
\newline
وقتی قصد مرجوعی کالای خریداری شده را دارم
\newline
باید امکان تماس با تیم پشتیبانی فروش را داشته باشم.

\item سناریو : کاربری که با قصد مرجوعی کالای خریداری شده با تیم پشتیبانی فروش تماس میگیرد، تایید تیم پشتیبانی فروش را دریافت میکند.
\newline
به عنوان کاربری که با قصد مرجوعی کالای خریداری شده با تیم پشتیبانی فروش تماس میگیرد
\newline
وقتی با تیم پشتیبانی فروش تماس می گیرم، در صورت مهیا بودن شرایط
\newline
باید تایید تیم پشتیبانی فروش را دریافت کنم.

\item سناریو : کاربری که با واحد پشتیبانی هماهنگی درخصوص مرجوع کردن محصول را انجام داده، محصول را به پیک موتوری تحویل می دهد.
\newline
به عنوان  کاربری که قصد برگشت دادن کالای خود را دارد
\newline
وقتی میخواهم کالای خود را برگشت دهم
\newline
باید آن را به پیک موتوری تحویل دهم.

\item سناریو : کاربری که کالای خود را بازگردانده است، مبلغ کالای بازگشتی به او بازگردانده شود.
\newline
به عنوان کاربری که کالای خود را بازگردانده است
\newline
وقتی کالای خود برگشت می دهم
\newline
باید مبلغ کالای بازگشتی را دریافت کنم.


\end{itemize}
\end{flushright}
\end{document}
