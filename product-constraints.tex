
 \documentclass[14pt]{article}
\usepackage[a4paper, left=1.5cm, right=1.5cm, top=1.5cm, bottom=1.5cm]{geometry}

\usepackage{xepersian} 

\settextfont{XB Zar}

\begin{document}


\section{ محدودیت‌های محصول(خدمت)}
محدودیت‌های فروشگاه آنلاین : 
زمان تحویل محصولات فیزیکی: در خرید حضوری می توان محصول را در لحظه دریافت کرد. اما این امر در خرید آنلاین اتفاق نمی افتد. از فروشگاه های آنلاین غالباً برای خرید کالاهایی استفاده می شود که به صورت محلی در دسترس افراد نیستند؛ در نتیجه برمبنای مورد، مدتی  بیشتر از زمان خرید حضوری طول می کشد تا سفارش به دست مشتری برسد.\\
عدم اطمینان به محصول فیزیکی، تأمین کننده و تحویل: در خرید حضوری، با توجه به امکان مشاهده و تعامل با محصول مشتری از ماهیت و کیفیت محصول اطلاع دارد و نگرانی از بابت دریافت محصول ندارد؛ زیرا محصول در اختیار اوست اما در خرید  الکترونیکی به این صورت نیست. دلیل این امر این است که اولاً، در شرایط عدم دسترسی فیزیکی به محصول، مشتری خرید را با انتظاراتی از ماهیت و کیفیت محصول انجام می دهد. ثانیا، از آنجا که مدیریت کسب و کارهای آنلاین می تواند در سراسر جهان انجام شود، آیا مشتری می تواند به قانونی بودن آن کسب و کار اطمینان داشته باشد یا خیر و اینکه آنها صرفا پول ما را نمی دزدند. درصورت بروز تخلف و دردی، شکایت یا جستجوی راه حقوقی در محاکمه آنها بسیار دشوار است! ثالثاً، حتی اگر جنس موردنظر ارسال شود، آیا مشتری می تواند به راحتی اطمینان داشته باشد که محصول به مقصد می رسد یا خیر.\\
کالاهای فاسدشدنی. کسب و کار های آنلاینن میتوانند برای تحویل سفارشات غذایی از حمل و نقل تخصصی یا یخچالی استفاده کنند اما کالاهایی که از طریق اینترنت خریداری و فروخته می شوند، دارای دوام و فساد ناپذیری هستند؛ زیرا آنها باید در مسیر از تامین کننده به کسب  و کار خریدار یا مصرف کننده سالم بمانند. این محدودیت های کالاهای فاسدشدنی یا بدون دوام، خرید این کالا به سمت کسب و کار های سنتی یا فروش و توزیع محلی کسب و کار های آنلاین سوق می دهند. در مقابل، کالاهای بادوام می توانند تقریباً از هر کس تا تقریباً هر شخص دیگری خریداری شود و سبب رقابت با قیمت های پایین تر می شود که مصرف کنندگان و سایر مشاغل که به دنبال خرید مستقیم تر از تولید کنندگان هستند که منجر به حذف واسطه گری می شود.\\
اطلاعات حسی محدود و انتخاب شده. اینترنت امکانات کافی برای حواس ما فراهم نمی کند. افراد صرفا در اینترنت میتوانند عکس محصولات را مشاهده کنند اما نمیتوانند وزن آنها را درک کنند. بعلاوه، وقتی در خرید حضوری می توان تمام اجزا کالا را بررسی کرد اما در اینترنت اینگونه نیست. در هنگام بررسی کالا ها از طریق اینترنت، تصویر هایی مشاهده می شود که فروشنده انتخاب کرده است اما ممکن است ویژگی های دیگر محصول مشاهده نشود و همچنین به عنوان مثال، نمی توان اتومبیل هنگام تغییر دنده را تست کرد یا احساس بو و احساس صندلی های چرمی منتقل شود و صدای موتور شنیده شود. این کمبود اطلاعات حسی سبب می شود که خرید اجناسی مانند کالاهای عمومی - چیزهایی که قبلاً دیده یا تجربه کرده اند و در مورد آنها ابهام کمی وجود دارد (نه چیزهای منحصر به فرد یا پیچیده) برای مردم راحت تر باشد.\\
برگشت کالا. بازگشت کالا بصورت آنلاین می تواند زمینه ساز مشکل باشد. عدم قطعیت مربوط به بازگشت پرداخت اولیه و تحویل کالا می تواند در این روند تشدید شود. همچنین افراد با چالش های دیگری روبرو میشوند : آیا کالاها به مبدا خود برمی گردند؟ چه کسی هزینه پست برگشتی را پرداخت می کند؟ آیا مبلغ کالا بازگردانده می شود؟ چقد طول میکشد؟ حال با مقایسه این چالش ها با تجربه بازگرداندن کالا در مغازه میتوان به مشکلات آنها پی برد.\\
حریم خصوصی، امنیت، پرداخت، هویت. کلاهبرداری در خرید آنلاین بزرگترین عیب خرید آنلاین است. در واقع درگاه های خرید اینترنتی جعلی برخی از محصولات عالی را در وب سایت به نمایش می گذارند و مشتریان را به خرید محصول جلب می کنند که منجر به سو استفاده از اطلاعات خصوصی، جزئیات پرداخت (به عنوان مثال جزئیات کارت اعتباری) و سرقت هویت می شود و همچنین چالش های انتخاب قوانین قضایی برای فراهم کردن امنیت اطلاعات افراد نیز وجود دارد. \\
خدمات و موارد غیر منتظره. کسب و کار های الکترونیکی وسیله ای موثر برای مدیریت تراکنش های شناخته شده و خدمات انجام شده مانند کار های روزمره هستند اما برای برخورد با موارد جدید یا غیرمنتظره مناسب نیستند. به عنوان مثال، یک شرکت حمل و نقل با کار کردن بسته های ساده عادت دارد اما نمیتواند درخواست انتقال یک بسته خیلی بزرگ و سنگین را پشتیبانی کند.\\
خدمات شخصی. اگرچه برخی از تعاملات انسانی را می توان از طریق اینترنت تسهیل کرد اما کسب و کار های الکترونیکی نمی تواند تعامل با کیفیت توسط خدمات شخصی را ارائه دهد. برای بیشتر مشاغل، روش های کسب و کار الکترونیکی معادل یک انبار غنی از اطلاعات به جای فروشنده است. این همچنین به این معنی است که بازخورد درباره نحوه واکنش مردم به محصولات و خدمات نیز تمایل دارد که با استفاده از رویکردهای تجارت الکترونیکی جزئی تر باشد یا شاید از بین برود. اگر تنها بازخورد شما این است که مردم محصولات یا خدمات شما را به صورت آنلاین خریداری می کنند ، این برای ارزیابی نحوه تغییر یا بهبود استراتژی های تجارت الکترونیکی و / یا محصولات و خدمات شما کافی نیست. استفاده موفقیت آمیز از تجارت الکترونیکی به طور معمول شامل راهکارهایی برای به دست آوردن و استفاده از بازخورد مشتری است. این به کسب و کارها کمک می کند تا نیازها و ترجیحات آنلاین مشتری را تغییر دهند ، آن را پیش بینی کرده و برآورده کنند ، که به دلیل سرعت نسبتاً سریع تغییرات مبتنی بر اینترنت بسیار حیاتی است.\\
اندازه و تعداد معاملات. کسب و کار های الکترونیکی غالباً سعی دارند که پرداخت را آسان کنند اما حداقل فرایند هایی برای پرداخت وجود دارد؛ در نتیجه معاملات بسیار کوچک و بسیار بزرگ به صورت آنلاین انجام نمی شوند. اندازه معاملات نیز تحت تأثیر اقتصاد حمل و نقل کالاهای فیزیکی قرار دارد. به عنوان مثال ، هرگونه مزیت یا سهولت خرید آنلاین یک محصول از کسب و کارهای مستقر در ایالات متحده با هزینه پرداخت هزینه برای تحویل به شما در استرالیا تحت الشعاع قرار می گیرد. هزینه تحویل همچنین به این معنی است که خرید یک کالا از کسب و کار های خارج از کشور بسیار گران تر از خرید همه کالاها از یک کسب و کار خارج از کشور است زیرا کالاها را می توان باهم بسته بندی و حمل کرد.\\
چانه زنی: فقط در خرید حضوری برخلاف خرید آنلاین می توان چانه زد." در خرید آنلاین ، وجه نقد ، تخفیف و کوپن دریافت می کنید اما این همان معامله نیست. اگر کسی در چانه زنی مهارت دارد ، به من اعتماد کنید ، او می تواند در خریدهای آفلاین مقدار زیادی پس انداز کند." باشه یا نه؟\\
هزینه های پنهان و هزینه حمل و نقل : هنگامی که برای اولین بار مبلغ محصول را در پورتال مشاهده می شود، به طور کلی ارزان تر به نظر می رسد. اما هنگامی که مبلغ پرداختی محاسبه میشود، هزینه های اضافی مانند هزینه حمل و نقل، مالیات و هزینه بسته بندی اضافه می شود. این هزینه ها محصول را نسبت به فروشگاه محلی گران می کند. در صورت خرید بیش از مقدار مشخص، برخی از درگاه ها حمل و نقل رایگان را ارائه می دهند. بعضی اوقات فقط برای استفاده رایگان از حمل و نقل، شما بیش از نیاز خود خرید می کنید.\\
تعامل : در صورت خرید حضوری، فروشندگان جزئیات کاملی از محصول را ارائه می دهند و میتوان از آنها سوال پرسید اما این امر در مورد خرید آنلاین اتفاق نمی افتد. تنها با دیدن تصویر و خواندن توضیحات میتوان خرید را انجام داد. برخی از پورتال ها به خریداران اجازه می دهند تا بررسی مشتری را هم بررسی کنند اما کافی نیست.
\end{document}