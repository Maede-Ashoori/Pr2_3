\documentclass[12pt]{report}
\usepackage{xepersian}
\settextfont{XB Zar}

\begin{document}
\section{تنظیم جدول فرایندهای انتخاب و ثبت سفارش}
اکتورها : مشتری ثبت نام شده - سامانه\\
نام مورد کاربرد : ایجاد سفارش در سامانه\\
توضیح : کاربر ثبت نام شده پس از ورود به حساب کاربری، فروشگاه مدنظر خود را از بین فروشگاه هایی که برای او نمایش داده شده، انتخاب مینماید و لیست کالاهای موجود در آن را مشاهده میکند. سپس در یک سبد خرید، محصولات مورد نیاز خود را به همراه تعداد هر کدام در سامانه وارد میکند.سپس سیستم لیست کالاهای سفارش مشتری را بررسی مینماید تا مشکلی در آن وجود نداشته باشد.(از نظر موجودبودن تمامی کالاهای درخواستی در فروشگاه مدنظر و یا قرارداشتن زمان سفارش در محدوده زمان کاری فروشگاه )\\
گام ها :\\
1- ورود به سامانه\\
2- ورود به حساب کاربری\\
3- مشاهده لیست فروشگاه های لباس نزدیک و انتخاب فروشگاه\\
4- مشاهده لیست کالاهای موجود در فروشگاه\\
5- انتخاب کالای مورد نظر\\
6-اضافه کردن نام و تعداد محصولات به سبد خرید و تکمیل سبد خرید\\
7- تایید لیست کالاهای سفارش توسط سامانه\\
سناریو جایگزین :\\
:7-a لیست کالاهای سفارش به علت عدم موجودی کافی آن کالا در فروشگاه رد شود. در این حالت  مشتری باید لیست خرید خود را ویرایش کند.\\
:7-b لیست کالاهای سفارش به علت خارج بودن زمان سفارش از محدوده زمانی کاری فروشگاه رد شو.در این حالت مشتری باید در زمان کاری فروشگاه مجددا سفارش را ایجاد کند.
 

\end{document}