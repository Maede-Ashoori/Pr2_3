\documentclass[12pt]{report}
\usepackage{xepersian}
\settextfont{XB Zar}

\begin{document}
\section{شناسایی ذینفعان}
ذینفعان یعنی افراد و گروه‌هایی که به نوعی در موفقیت یا شکست یک کسب و کار، سهیم هستند که برای  کسب و کار مدنظر شامل موارد زیر می باشند:\\
۱- مشتریان کسب و کار‌ (بررسی شده در فایل دیگر)\
مشتریان برای تهیه کالا یا خدمات به شرکت وابسته هستند. آنها با هر خریدی که انجام می دهند از کسب وکار پشتیبانی می کنند و همچنین هر خرید آن ها به کسب وکار نشان می دهد که در چه محصولات و خدماتی سرمایه گذاری بیشتری انجام دهد. با انجام این کار، مشتریان به راهنمایی جهت یک تجارت کوچک کمک می کنند.\\
۲- صاحب کسب وکار(مدیرعامل)\\
صاحب کسب وکار کنترل نهایی بر شرکت دارد و کلیه تصمیمات راهبردی و تصمیم گیری نهایی در ارتباط با تفویض وظایف و … برعهده اوست.\\
۳- تیم IT \\
اعضای تیم که در راه اندازی سایت و اپلیکیشن کسب وکار وخدمات مربوط به آن و کلیه خدمات مرتبط به فناوری اطلاعات نقش دارند.
مدیر اجرایی IT  که نقش مهمی در پروژه تجارت الکترونیکی و رهبری آن دارد و نگرانی های فناوری اطلاعات را به مجموعه منتقل می کند. \\
۴- تیم فروش و بازاریابی\\
تیم فروش نقش به سزایی در موفقیت کسب وکار ایفا میکند؛ چرا که در واقع کسب کارآمد درآمد به این گروه وابسته است. در ارتباط با کسب وکارهای آنلاین استفاده مناسب از فناوری ها و سیستم های اطلاعاتی همچون نرم افزارهای مدیریت ارتباط با مشتریان \footnote{\label{myfootnote}CRM} و زیرساخت های مناسب به بهبود عملکرد آن ها و کسب درآمد و سود بیشتر کمک میکند.\\
تیم بازاریابی (و به صورت خاص دیجیتال مارکتینگ)\\
اهمیت بازاریابی جهت تولید ارزش برای کسب وکار برکسی پوشیده نیست. این امر به خصوص در اکوسیستم امروزی کسب وکارها و در مورد کسب و کارهای آنلاین اهمیت بیشتری می یابد. به طور کلی میتوان گفت که محتوای پیشرفته می تواند مشتریان بالقوه را تا 10٪ افزایش دهد.
این تیم مسئولیت ایجاد تجربه کاربری جذاب برای مشتریان را برعهده دارد. وظایف آن ها شامل ایجاد محتوای جذاب برای رسانه های مختلف، ایجاد کمپین های مختلف، تخفیفات و ارائه امکانات خاص به مشتریان، مدیریت باشگاه مشتریان و گزارش درباره فعالیت های یکپارچه بازاریابی دیجیتال می باشد . در این راستا استفاده از نرم افزارها و سامانه های اطلاعاتی همچون CRM موثر است.\\
۵- تیم حسابداری و مالی \\
این تیم مدیریت کلیه جریان های مالی و قوانین مرتبط با رگولاتوری و تبعیت از قوانین مالیاتی و … را بر عهده دارند.انجام درست وظایف این تیم ،‌سلامت مالی کسب و کار و به تبع آن موفقیت آن را تضمین می کند.\\
۶- فروشگاه هایی که با شرکت قرارداد امضا کرده و همکاری می کنند. 
شرکت برای فروش محصولات با فروشگاه ها قرارداد می بندد و هرچه سود ناشی از فروش محصولات در پلتفرم آنلاین بیشتر باشد، سود فروشگاه ها نیز افزایش می یابد. به بیان دیگر، بعضی از فروشگاه ها فروشی غیر از فروش آنلاین در پلتفرم مذکور ندارند؛ درنتیجه تنها راه کسب درآمد آنها فروش در پلتفرم آنلاین میباشد.\\
۷- پیک های موتوری \\
حمل و نقل روان کالا برای کسب و کار های آنلاین ضروری است؛ زیرا تحویل به موقع و موثر میتواند تجربه مشتری را بهبود دهد که این مسئله به افزایش سود منجر می شود. از طرف دیگر، هر چه درآمد شرکت افزایش یابد، درآمد پیک های موتوری هم می تواند افزایش پیدا کند.\\
۸- سرمایه گذار\\
حمایت سرمایه گذاران از کسب وکار به کمک سرمایه ای که در اختیار یک کسب و کار میگذارند،‌نقش به سزایی در شکل گیری اولیه کسب وکار و در ادامه برای حیات آن کسب وکار دارد. حضور سرمایه گذاران خوب میتواند مسیر موفقیت کسب و کار را به کلی تغییر دهد.\\
۹- رقبا\\
شرایط رقابتی در میزان سود و زیان و هم چنین تلاش برای بقای شرکت ها بسیار موثر است. رقبا میتواند به عنوان تلنگری برای تلاش بیشتر یا اهرم فشار یک کسب و کار باشند و نقاط ضعف رقبا میتواند به عنوان یک عامل سود آور برای کسب و کار تلقی شود و برعکس این نیز صادق است. با توجه به این مسئله، رقبا نقش بسیار بزرگی در موفقیت یا عدم موفقیت کسب و کار دارند.\\
۱۰- ذینفعان مرتبط با امور پرداخت\\
ذینفعان مختلفی در هر پرداخت آنلاین دخیل هستند. هر یک از آنها نقش خود را دارند و جریان معاملات به روش زیر انجام می شود:\\
-بانک صادرکننده (فرستنده) کارت بانکی را صادر می‌کند و بنابراین این روش پرداخت را برای خریدار و فرصت فروش را برای شرکت امکان پذیر می کند. همچنین این ذینفع اطلاعات کارت پرداخت و دارنده کارت را تأیید می کند. \\
-درگاه پرداخت اینترنتی، تراکنش را پردازش می کند ، یعنی آن را از شرکت به بانک گیرنده هدایت می کند تا معامله پرداخت را آغاز کند. همچنین دروازه پرداخت اینترنتی این پاسخ را از بانک گیرنده به شرکت منتقل می کند تا نتیجه تراکنش را گزارش کند. به شرطی که تراکنش مجاز باشد ، بانک ها حساب خود را تسویه می کنند.\\
-بانک گیرنده که مقصد نهایی تراکنش است.\\
بانک ها دلیل مهمی برای ادامه فعالیت های اینترنتی دارند . زیرا تجارت با گذر زمان به سمت الکترونیک سوق پیدا می کند. در این صورت ، آنها پرداختی را برای خریداران و فروشندگان مشغول تجارت الکترونیکی پردازش می كنند. \\
درگاه های تجارت الکترونیکی بسترهای اینترنتی هستند که در آن معاملات تجاری خریدار و فروش انجام می شود. پورتال تجارت الکترونیکی عمدتا توسط تولید کنندگان و توزیع کنندگان بزرگ برای انتقال خریداران خود از فروش حضوری سنتی به یک کانال خرید آنلاین مورد استفاده قرار می گیرد. برای موفقیت، درگاه های تجارت الکترونیکی باید تجربه شخصی شده ای را در اختیار خریداران قرار دهند و سفارش عمده را تسهیل کنند.
\end{document}