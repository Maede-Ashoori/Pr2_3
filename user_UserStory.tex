\documentclass[12pt]{report}
\usepackage{xepersian}
\settextfont{XB Zar}

\begin{document}
\section{ شناسایی کاربران محصول جهت تکمیل داستان کاربری}
کاربر ثبت نام نشده : کاربری که جدید و به صورت مهمان وارد سایت شده است و یا عملیات ثبت نام او با موفقیت انجام نشده است.\\
کاربر ثبت نام شده : کاربری که شماره همراه (یا ایمیل) خود را در سامانه ثبت نام ثبت کرده است و با وارد کردن رمز تایید در پیامک دریافتی (یا با وارد شدن از طریق لینک احراز هویت ایمیل) ثبت نام خود را تکمیل کرده است.\\
کاربری که قصد نهایی کردن سفارش خود را دارد : کاربری که پس از انتخاب محصول مورد نظر خود و اضافه کردن آن به سبد خرید، وارد صفحه درگاه بانکی برای پرداخت می شود.\\
کاربری که اطلاعات کارت بانکی خود را ثبت کرده است : کاربری که در صفحه درگاه بانکی به ثبت اطلاعات کارت بانکی می پردازد.\\
کاربری که در مرحله ثبت نهایی سفارش در درگاه انتقال پرداخت است : کاربری که در صفحه درگاه بانکی اطلاعات کارت بانکی را نادرست وارد کرده است یا موجودی کافی برای خرید ندارد.\\
کاربری که واریز خود را با موفقیت انجام داده است :  کاربری که اطلاعات کارت بانکی را درست ثبت کرده است و از حساب بانکی او مبلغ سفارش کم شده است. به بیان دیگر، عملیات پرداخت وی تکمیل شده است. \\
کاربری که پرداخت او ناموفق بوده است : کاربری که پرداخت مبلغ سفارش او ناموفق بوده است و به به لیست خرید خود بازگشت داده شده است.\\
کاربری که سفارش خود را دریافت کرده است : کاربری که فرایند خرید او با موفقیت انجام شده است و سفارشش را دریافت نموده است.\\
کاربری که قصد مرجوعی کالای خریداری شده را دارد : کاربری که برای پس دادن کالای خود با تیم پشتیبانی فروش ارتباط برقرار می کند.\\
کاربری که با قصد مرجوعی کالای خریداری شده با تیم پشتیبانی فروش تماس میگیرد : کاربری که قصد دارد کالای خریداری شده خود را بازگرداند که به این منظور با تیم پشتیبانی فروش تماس می گیرد.\\
کاربری که کالای خود را بازگردانده است : زمانی که کاربر کالای خود را بازمیگرداند و مبلغ کالا به حساب وی واریز می شود.\\
*فرضیات: \\
کاربران برای ثبت نام، فرم ثبت نام را مشاهده می کنند.\\
کاربران بدون ثبت نام نمیتوانند لیست فروشگاه های نزدیک به خود را مشاهده کنند.\\
کاربران برای ثبت نام به وارد کردن رمز تایید در پیامک یا ورود به لینک احراز هویت در ایمیل نیاز دارند.\\
پرداخت مبلغ سفارش با استفاده از رمز پویا و ارسال پیامک رمز پویا از طرف بانک انجام می شود.\\
زمانی کاربر کالای خود را بازمیگرداند، برگشت مبلغ کالا به حساب وی واریز میشود.\\
\end{document}