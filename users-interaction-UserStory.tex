\documentclass[12pt]{report}
\usepackage{xepersian}
\settextfont{XB Zar}

\begin{document}
\section{ شناسایی روش های مختلف استفاده کاربران از سیستم جهت تکمیل داستان کاربری}
شناسایی روش های مختلف استفاده کاربران از سیستم:\\
کاربران از طریق فرمی که سایت نشان میدهد، برای ثبت نام اقدام میکنند.\\
کاربران با ثبت اطلاعات شماره همراه یا ایمیل، پیامک رمز تایید یا ایمیل احراز هویت دریافت می کنند و از طریق رمز تایید یا لینک در ایمیل احراز هویت عملیات ثبت نام با موفقیت انجام می شود.\\
برای ایجاد سفارش، کاربران لیستی از فروشگاه های لباس نزدیک به آن ها و هزینه ارسال سفارش از طریق پیک را مشاهده می کنند.\\
زمانی که کاربران ثبت نام میکنند و فروشگاه مدنظر خود را انتخاب می نماید و لیست کالاهای موجود در آن را مشاهده می کنند. \\
کاربران پس از مشاهده لیست کالاهای فروشگاه موردنظر، محصولات مدنظر خود را به سبد خرید اضافه میکنند.\\
وقتی که کاربران محصول مورد نظر را به سبد خرید خود اضافه می کنند، تایید سیستم را به منظور تایید لیست کالاهای مورد انتخابی دریافت می کنند.\\
کاربران پس از ایجاد کردن سبد خرید، ممکن است اخطار سیستم را به منظور رد لیست کالاهای انتخابی ( به علت موجود نبودن کالا یا سفارش خارج از محدوده زمان کاری فروشگاه) دریافت کنند.\\
کاربران برای نهایی کردن سفارش و واریز مبلغ سفارش خود، به صفحه درگاه بانکی انتقال داده می شوند.\\
زمانی که کاربران اطلاعات کارت بانکی خود وارد کردند، برای وارد کردن رمز پویا، پیامک رمز پویا ارسال میشود.\\
ممکن است کاربران اطلاعات کارت بانکی یا رمز پویا را نادرست وارد کنند یا موجودی کافی در حساب شخصی خود نداشته باشد، پس سیستم اخطاری به همراه علت مرتبط ارسال میکند.\\
کاربرانی که واریز خود را انجام داده باشند، تایید موفقیت پرداخت را از سیستم دریافت می کنند.\\
ممکن است پرداخت مبلغ سفارش کاربری ناموفق باشد، در نتیجه آن کاربر به لیست خرید خود بازگشت داده می شود و امکان مشاهده مجدد لیست و اضافه یا کم کردن محصولات از لیست مهیا می باشد.\\
زمانی که کاربری سفارش خود را دریافت کرده باشد، امکان به اشتراک گذاشتن نظرات خود را دارد و می تواند فرم ثبت نظر و امتیاز را مشاهده کند.\\\
کاربرانی که قصد مرجوعی کالای خریداری شده را دارند می توانند با تیم پشتیبانی فروش ارتباط برقرار کنند و مسئله خود را مطرح کنند.\\
زمانی که کاربران با قصد مرجوعی کالای خریداری شده با تیم پشتیبانی فروش تماس می گیرند، باید در صورت مهیا بودن شرایط، تایید تیم پشتیبانی فروش را دریافت کنند تا بتوانند کالای خود را برگردانند.\\
کاربرانی که با واحد پشتیبانی هماهنگی های لازم را درخصوص مرجوع کردن محصول انجام داده اند، میتوانند محصول را به پیک موتوری تحویل دهند.\\
زمانی که کاربر کالای خود را بازمیگرداند، مبلغ کالا به حساب وی واریز می شود.\\
\end{document}

