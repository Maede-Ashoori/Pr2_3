\documentclass[12pt]{report}
\usepackage{xepersian}
\settextfont{XB Zar}

\begin{document}
\section{چشم اندازهای مختلف ذینفعان و حوزه های ارزش آنان}
چشم انداز مشتری\\
مشتریان از اصلی ترین ذینفعان هر کسب وکار و به ویژه کسب و کار مدنظر می باشند.دغدغه این گروه از ذینفعان در رابطه با محصولاتی که خریداری می کنند و یا خدماتی که دریافت می کنند، مرتبط با قیمت ، کیفیت ، در دسترس بودن ، تحویل و خدمات پس از فروش می باشد. همچنین شیوه برخورد کسب وکار با آن ها از اهمیت زیادی برخوردار است.یکی از راه های افزایش ارزش ذینفعان از دیدگاه مشتری تمرکز بر بازاریابی رابطه ای است. این رویکرد با استفاده از تحقیقات بازار، نیازها ، خواسته ها ، علایق و ارزش های مشتریان واقعی و بالقوه را تعیین می کند.\\
چشم انداز تأمین کنندگان\\
ارزش برای تامین کنندگان از طریق دستیابی به قیمت مناسب و بازار منظم کالاها و خدمات فراهم می شود.
روابط تأمین کننده می تواند از طریق پشتیبانی از فعالیت های تأمین کننده توسعه یابد. برخی از سازمان ها تأمین کنندگان خود را به طور فعال درگیر فرآیندهای ارتباطی می کنند (به عنوان مثال در مورد راه هایی که تأمین کننده به محصولات و خدمات سازمان کمک می کند) و برخی حتی بیشتر پیش می روند ، به تامین کنندگان خود مشاوره داده و کمک می کنند تا با مشکلات مدیریتی و عملیاتی خود مقابله کنند.عملکرد تأمین کنندگان با معیارهایی همچون سرعت تحویل یا سرعت پاسخ به الزامات غیر معمول سنجیده می شود. در کسب و کار مورد بحث تامین کنندگان همان فروشگاه های طرف قرارداد هستند که فروشگاه آنلاین را تشکیل می دهند.\\
 چشم انداز کارکنان\\
ارزش به کارکنان از طریق حقوق، دستمزد و مزایا ، امنیت شغلی ، رفتار کارفرما با آن ها (سبک مدیریتی) ، فرصت های آموزش ، توسعه و ارتقا، ، بهداشت و ایمنی پیرامون کار فراهم می شود.از اقدامات مهم درحوزه ارزش رساندن به کارکنان توجه مناسب به امر مدیریت سرمایه انسانی است. کارکنان کسب وکار مورد بحث، شامل تمامی تیم های عملیاتی همچون IT، فروش و بازاریابی، حسابداری و مالی می باشد.\\
 چشم انداز حسابرسی و سرمایه\\
حسابرسان و سرمایه گذاران با مدیریت بازرگانی و مالی سازمان در ارتباط هستند و ارزش به آنان از آن حوزه تامین می شود. بازتاب این امر در ساختار امور مالی از نظر مدیریت دارایی ، جریان های نقدی و بازده سرمایه گذاری و ساختار حاکمیت شرکتی دیده می شود. حاکمیت شرکت به نحوه اداره کسب وکار توسط مدیریت ارشد و به ویژه نحوه انتخاب تیم مدیریت عالی ، نحوه پاداش مدیران عالی و نحوه کنترل و حسابرسی امور مالی سازمان مربوط می شود.\\
چشم انداز دولت\\
دولت به توانایی سازمانها برای کمک به مالیات عمومی، اشتغال و رفاه جامعه علاقه مند است. دولت همچنین موظف است اطمینان حاصل کند که سازمانها به درستی اداره می شوند ، در فعالیتهایشان اخلاقی و مطابق با قوانین تنظیم کننده کسب و کار، اشتغال، رقابت، بهداشت و ایمنی و تهیه کالاها و خدمات عمل می کنند. \\
\end{document}