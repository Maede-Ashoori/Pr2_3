\documentclass[12pt]{report}
\usepackage{xepersian}
\settextfont{XB Zar}

\begin{document}
\section{شناسایی علت وجود محصول}
مردم در دنیای امروز برای خرید محصولات مورد نیاز خود به اینترنت مراجعه می کنند. آنها می توانند به راحتی از طریق رایانه های شخصی یا تلفن همراه خود به فروشگاه های آنلاین دسترسی پیدا کنند و آنچه را که می خواهند سفارش دهند. محصولات سفارش داده شده نیز به درب منزل تحویل داده می شوند. در ادامه لیستی از 7 دلیل آورده شده است که ثابت می کند چرا خرید آنلاین که در واقع همان خدمت مورد بررسی در این بخش است، در مقایسه با خرید حضوری در فروشگاه ها بهتر عمل می کند و به عبارتی علت پدید آمدن فروشگاه های آنلاین را مورد بررسی قرار می دهد:\\
1.صرفه جویی در وقت. زیرا نیازی به مراجعه به فروشگاه ها ندارد. برای خرید می توان از طریق رایانه یا موبایل به سادگی وارد وب سایت فروشگاه خرده فروشی شد و حتی می توان همزمان از چندین فروشگاه خرید کرد. از آنجا که خرید آنلاین می تواند به مردم کمک کند تا وقت گرانبهای خود را صرفه جویی کنند، بسیار مورد توجه قرار گرفته است.\\
2. کاهش هزینه های حمل و نقل.خرید آنلاین با هیچ هزینه حمل و نقل همراه نیست. میتوان به سادگی اجناس را از خانه سفارش داد. تمام محصولات نیز درب منزل تحویل داده می شوند. بنابراین از هزینه های حمل و نقل جلوگیری می شود.\\
3. عدم محدودیت زمانی برای خرید. بیشتر فروشگاه ها فقط در طول روز باز هستند. با این وجود، ممکن است به دلیل سایر تعهدات، در طول روز بیرون رفتن و خرید مقدور نباشد. در چنین شرایطی خرید آنلاین کمک خواهد کرد. فروشگاه های خرید آنلاین در 24 ساعت شبانه روز باز هستند. بنابراین، می توان در هر زمان مناسب خرید کرد.\\
4. قیمت کمتر محصولات. محصولات موجود در فروشگاه های آنلاین در مقایسه با فروشگاه های فیزیکی معمولاً ارزان تر هستند. از طرف دیگر، تخفیفات ویژه ای مانند تخفیف های مناسبتی نیز این امکان را فراهم می کند که مبلغ قابل توجهی در خرید را پس انداز کرد که امکان دریافت این چنین تخفیف های شگفت انگیزی از فروشگاه های فیزیکی کمتر است.\\
5- کاهش زمان انتظار.متأسفانه هنگام خرید کالاها در فروشگاه های آفلاین، امکان جلوگیری از صف وجود ندارد. اما در خرید آنلاین مفهوم صف وجود ندارد. فقط باید کالای موردنظر را به سبد خرید اضافه کرد و پرداخت کرد.\\
6. کاهش ازدحام جمعیت. فروشگاه های شلوغ هرگز تجربه خوشایندی را برای افرادی که خرید می کنند ایجاد نمی کند. در فروشگاه های خرید آنلاین شلوغی معنا ندارد. بنابراین خرید یک تجربه بهتر خواهد بود.\\
7. سهولت یافتن اجناس.در خرید آنلاین می توان به راحتی اجناس را جستجو کنید. فیلترهای زیادی نیز برای راحتی موجود است. بنابراین ، می توان به سرعت خرید کرد.
\end{document}
