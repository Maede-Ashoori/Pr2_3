
\documentclass[14pt]{article}
\usepackage{xepersian} 

\settextfont{XB Zar}
\begin{document} 
\section{شناسایی مشتریان}

فرضیات : 
\newline
\begin{flushright}
\begin{itemize}

\item این کسب و کار با فروشگاه هایی همکاری میکند که اجناس آن ها  بازه قیمتی متوسط و رو به بالا دارند.(پوشاک با سطح کیفیت متوسط و بالا و اجناس برند و لوکس)
رده سنی کاربران هدف، شامل تمام رده ها :‌کودک، نوجوان، جوان و بزرگسال بوده و برای هر دو جنسیت مونث و مذکر می باشد.
\item همکاری با فروشگاه های پوشاک در سرتاسر شهر تهران صورت میگیرد و محدود به همین شهر است.
\item برای یک فروشگاه آنلاین پوشاک دسته های مختلفی از کاربران (مشتریان) که با فروشگاه در تعاملند،‌ قابل تصور است. مشتریان کسب و کار فرضی مد نظر براساس فرضیات ذکرشده  به بخش های زیر تقسیم می شوند:
\item از نظر جنسیت: مخاطبان کسب و کار شامل هر دو جنسیت مونث و مذکر می باشند.
\item از نظر سن:‌ مخاطبان کسب و کار شامل همه رده های سنی (کودک و نوجوان، جوان  و بزرگسالان) می باشند . 
\item از نظر جغرافیایی: ساکنین تهران و شهرهای نزدیک که امکان ارسال کالا با پیک برای آن ها فراهم است (به عنوان مثال:‌کرج)
\item از نظر رفتاری ‌: مشتریانی که برای کیفیت خوب کالا ارزش قائلند و به علاوه اعتماد به خرید آنلاین دارند و توانایی انجام این نوع خرید در آن ها وجود دارد 
\end{itemize}

\end{flushright}
\end{document}
